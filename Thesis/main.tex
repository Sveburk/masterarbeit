%--------------------------------------------------------------------------------------%
%--------------------------------------------------------------------------------------%
%-----------------------------    S E T T I N G S     ---------------------------------%
%--------------------------------------------------------------------------------------%
%--------------------------------------------------------------------------------------%



% Options for packages loaded elsewhere
\PassOptionsToPackage{unicode}{hyperref}
\PassOptionsToPackage{hyphens}{url}
\PassOptionsToPackage{dvipsnames,svgnames,x11names}{xcolor}
%
\documentclass[
  12pt,
  a4paper,
]{article}


%--------------------------------------------------------------------------------------%
%-----------------------------    S E T T I N G S     ---------------------------------%
%-----------------------------      P A K E T E        --------------------------------%
%--------------------------------------------------------------------------------------%
\usepackage[utf8]{inputenc}
\usepackage{amsmath,amssymb}
\usepackage{xcolor}
\usepackage{afterpage}
\usepackage{hyperref}
\usepackage{graphicx}
\usepackage{subcaption}  % Für die Verwendung von subfigure
\usepackage{setspace}  % Für den Befehl \setstretch
\usepackage{transparent}
\usepackage{tikz}
\usepackage{eso-pic} % Für Hintergrundbilder
\usepackage[backend=biber]{biblatex}
\usepackage{csquotes}



\ifPDFTeX
  \usepackage[T1]{fontenc}
  \usepackage[utf8]{inputenc}
  \usepackage{textcomp} % provide euro and other symbols
\else % if luatex or xetex
  \usepackage{unicode-math}
  \defaultfontfeatures{Scale=MatchLowercase}
  \defaultfontfeatures[\rmfamily]{Ligatures=TeX,Scale=1}
\fi
\usepackage{lmodern}
\ifPDFTeX\else  
    % xetex/luatex font selection
\fi
% Use upquote if available, for straight quotes in verbatim environments
\IfFileExists{upquote.sty}{\usepackage{upquote}}{}
\IfFileExists{microtype.sty}{% use microtype if available
  \usepackage[]{microtype}
  \UseMicrotypeSet[protrusion]{basicmath} % disable protrusion for tt fonts
}{}
\makeatletter
\@ifundefined{KOMAClassName}{% if non-KOMA class
  \IfFileExists{parskip.sty}{%
    \usepackage{parskip}
  }{% else
    \setlength{\parindent}{0pt}
    \setlength{\parskip}{6pt plus 2pt minus 1pt}}
}{% if KOMA class
  \KOMAoptions{parskip=half}}
\makeatother
\usepackage{xcolor}
\usepackage[lmargin=2.5cm,rmargin=2.5cm,tmargin=2cm,bmargin=2cm]{geometry}
\setlength{\emergencystretch}{3em} % prevent overfull lines
\setcounter{secnumdepth}{-\maxdimen} % remove section numbering
% Make \paragraph and \subparagraph free-standing
\makeatletter
\ifx\paragraph\undefined\else
  \let\oldparagraph\paragraph
  \renewcommand{\paragraph}{
    \@ifstar
      \xxxParagraphStar
      \xxxParagraphNoStar
  }
  \newcommand{\xxxParagraphStar}[1]{\oldparagraph*{#1}\mbox{}}
  \newcommand{\xxxParagraphNoStar}[1]{\oldparagraph{#1}\mbox{}}
\fi
\ifx\subparagraph\undefined\else
  \let\oldsubparagraph\subparagraph
  \renewcommand{\subparagraph}{
    \@ifstar
      \xxxSubParagraphStar
      \xxxSubParagraphNoStar
  }
  \newcommand{\xxxSubParagraphStar}[1]{\oldsubparagraph*{#1}\mbox{}}
  \newcommand{\xxxSubParagraphNoStar}[1]{\oldsubparagraph{#1}\mbox{}}
\fi
\makeatother


\providecommand{\tightlist}{%
  \setlength{\itemsep}{0pt}\setlength{\parskip}{0pt}}\usepackage{longtable,booktabs,array}
\usepackage{calc} % for calculating minipage widths
% Correct order of tables after \paragraph or \subparagraph
\usepackage{etoolbox}
\makeatletter
\patchcmd\longtable{\par}{\if@noskipsec\mbox{}\fi\par}{}{}
\makeatother
% Allow footnotes in longtable head/foot
\IfFileExists{footnotehyper.sty}{\usepackage{footnotehyper}}{\usepackage{footnote}}
\makesavenoteenv{longtable}
\usepackage{graphicx}
\makeatletter
\def\maxwidth{\ifdim\Gin@nat@width>\linewidth\linewidth\else\Gin@nat@width\fi}
\def\maxheight{\ifdim\Gin@nat@height>\textheight\textheight\else\Gin@nat@height\fi}
\makeatother
% Scale images if necessary, so that they will not overflow the page
% margins by default, and it is still possible to overwrite the defaults
% using explicit options in \includegraphics[width, height, ...]{}
\setkeys{Gin}{width=\maxwidth,height=\maxheight,keepaspectratio}
% Set default figure placement to htbp
\makeatletter
\def\fps@figure{htbp}
\makeatother
% definitions for citeproc citations
\NewDocumentCommand\citeproctext{}{}
\NewDocumentCommand\citeproc{mm}{%
  \begingroup\def\citeproctext{#2}\cite{#1}\endgroup}
\makeatletter
 % allow citations to break across lines
 \let\@cite@ofmt\@firstofone
 % avoid brackets around text for \cite:
 \def\@biblabel#1{}
 \def\@cite#1#2{{#1\if@tempswa , #2\fi}}
\makeatother
\newlength{\cslhangindent}
\setlength{\cslhangindent}{1.5em}
\newlength{\csllabelwidth}
\setlength{\csllabelwidth}{3em}
\newenvironment{CSLReferences}[2] % #1 hanging-indent, #2 entry-spacing
 {\begin{list}{}{%
  \setlength{\itemindent}{0pt}
  \setlength{\leftmargin}{0pt}
  \setlength{\parsep}{0pt}
  % turn on hanging indent if param 1 is 1
  \ifodd #1
   \setlength{\leftmargin}{\cslhangindent}
   \setlength{\itemindent}{-1\cslhangindent}
  \fi
  % set entry spacing
  \setlength{\itemsep}{#2\baselineskip}}}
 {\end{list}}
\usepackage{calc}
\newcommand{\CSLBlock}[1]{\hfill\break\parbox[t]{\linewidth}{\strut\ignorespaces#1\strut}}
\newcommand{\CSLLeftMargin}[1]{\parbox[t]{\csllabelwidth}{\strut#1\strut}}
\newcommand{\CSLRightInline}[1]{\parbox[t]{\linewidth - \csllabelwidth}{\strut#1\strut}}
\newcommand{\CSLIndent}[1]{\hspace{\cslhangindent}#1}

\ifpdf
  \usepackage{authblk} % Paket für Affiliations
  \usepackage{orcidlink} % Paket für ORCID-Link
  \usepackage{pdfpages} % Paket zum Einbinden von PDFs
\makeatletter
\@ifpackageloaded{caption}{}{\usepackage{caption}}
\AtBeginDocument{%
\ifdefined\contentsname
  \renewcommand*\contentsname{Inhaltsverzeichnis}
\else
  \newcommand\contentsname{Inhaltsverzeichnis}
\fi
\ifdefined\listfigurename
  \renewcommand*\listfigurename{Abbildungsverzeichnis}
\else
  \newcommand\listfigurename{Abbildungsverzeichnis}
\fi
\ifdefined\listtablename
  \renewcommand*\listtablename{Tabellenverzeichnis}
\else
  \newcommand\listtablename{Tabellenverzeichnis}
\fi
\ifdefined\figurename
  \renewcommand*\figurename{Abbildung}
\else
  \newcommand\figurename{Abbildung}
\fi
\ifdefined\tablename
  \renewcommand*\tablename{Tabelle}
\else
  \newcommand\tablename{Tabelle}
\fi
}
\@ifpackageloaded{float}{}{\usepackage{float}}
\floatstyle{ruled}
\@ifundefined{c@chapter}{\newfloat{codelisting}{h}{lop}}{\newfloat{codelisting}{h}{lop}[chapter]}
\floatname{codelisting}{Listing}
\newcommand*\listoflistings{\listof{codelisting}{Listingverzeichnis}}
\makeatother
\makeatletter
\makeatother
\makeatletter
\@ifpackageloaded{caption}{}{\usepackage{caption}}
\@ifpackageloaded{subcaption}{}{\usepackage{subcaption}}
\makeatother

\ifLuaTeX
\usepackage[bidi=basic]{babel}
\else
\usepackage[bidi=default]{babel}
\fi
\babelprovide[main,import]{ngerman}
% get rid of language-specific shorthands (see #6817):
\let\LanguageShortHands\languageshorthands
\def\languageshorthands#1{}
\ifLuaTeX
  \usepackage{selnolig}  % disable illegal ligatures
\fi
\usepackage{bookmark}

\IfFileExists{xurl.sty}{\usepackage{xurl}}{} % add URL line breaks if available
\urlstyle{same} % disable monospaced font for URLs
\hypersetup{
  pdftitle={Masterarbeit Sven Burkhardt},
  pdfauthor={Sven Burkhardt},
  pdflang={de},
  colorlinks=true,
  linkcolor={blue},
  filecolor={Maroon},
  citecolor={Blue},
  urlcolor={Blue},
  pdfcreator={LaTeX via pandoc}
  }


\title{Masterarbeit}
\usepackage{etoolbox}
\makeatletter
\providecommand{\subtitle}[1]{% add subtitle to \maketitle
  \apptocmd{\@title}{\par {\large #1 \par}}{}{}
}
\makeatother
\subtitle{Hier könnte ihre Werbung stehen}
\author{Sven Burkhardt}
\date{2024-09-16}

%--------------------------------------------------------------------------------------%
%--------------------------------------------------------------------------------------%
%----------------------------- T I T E L B L A T T   ----------------------------------%
%--------------------------------------------------------------------------------------%
%--------------------------------------------------------------------------------------%
\begin{document}


% Setzt die Schriftfarbe auf Weiß
\color{white}
\pagecolor[HTML]{46505A } %Seitenfarbe in Uni Basel Anthrazit
\afterpage{\nopagecolor}  % Rückkehr zum Standardhintergrund auf den folgenden Seiten
\pagenumbering{gobble}    % Verhindert die Anzeige der Seitennummer auf dem Titelblatt


\date{} %killt Date
\author{} %killt Author
\maketitle


\begin{center}
  \author{\LARGE{\author{Sven Burkhardt}}}\\
  \vspace{3mm}
  \orcidlink{0009-0001-4954-4426} {0009-0001-4954-4426}\\ % Orcid Link und Nummer
  \begin{figure}[h]
    \centering
    \color{white}
    \href{https://dhlab.philhist.unibas.ch/en/persons/sven-burkhardt/}{\includegraphics[height=3.0
  mm]{./assets/Logos/Uni_basel_logo_white.png}} 17-056-912\\ %logo Unibas + Link + Immatrikulationsnummer
    \vspace{3mm}
    \large{2024-08-05}
  \end{figure}
  \setcounter{figure}{0}
\end{center}


% ------------ Hexagon grafik beginn -----------
\centering
\AddToShipoutPictureBG*{%
    \put(0,-40){%
        \includegraphics[width=\paperwidth]{./assets/Logos/Hexagon_Deko}
    }
}
\centering
\AddToShipoutPictureBG*{%
    \put(0,810){%
        \includegraphics[width=\paperwidth]{./assets/Logos/Hexagon_Deko}
    }
}
\centering
\AddToShipoutPictureBG*{%
    \put(33,752){%
        \includegraphics[width=\paperwidth]{./assets/Logos/Hexagon_Deko}
    }
}
\centering
\AddToShipoutPictureBG*{%
    \put(-99,752){%
        \includegraphics[width=\paperwidth]{./assets/Logos/Hexagon_Deko}
    }
}



\noindent % Verhindert Einzug des nachfolgenden Textes

% ------------ Hexagon grafik ende -----------



\begin{center}
    \vfill
    \begin{figure}
        \centering
        \begin{subfigure}{.3\textwidth}
          \centering
          \includegraphics[width=.8\linewidth]{./assets/Logos/uni-basel-logo-en_white.png}
        \end{subfigure}%
        \begin{subfigure}{.3\textwidth}
          \centering
          \includegraphics[width=.8\linewidth]{./assets/Logos/dhlab-logo-white.png}
        \end{subfigure}
        \end{figure}
        \setcounter{figure}{0}

    University of Basel\\
    Digital Humanities Lab\\
    Switzerland
\end{center}

\newpage
\newpage
\pagenumbering{arabic}
\color{black}           % Setzt die Schriftfarbe auf Schwarz für die folgenden Seiten
\setstretch{1.5}
\thispagestyle{empty}

%________________



%--------------------------------------------------------------------------------------%
%--------------------------------------------------------------------------------------%
%-----------------------------   A B S T R A C T     ----------------------------------%
%--------------------------------------------------------------------------------------%
%--------------------------------------------------------------------------------------%


\section*{Abstract}

Diese Arbeit befasst sich mit dem Archiv des Männerchor Murg in den Jahren der Weimarer Republik bis zum Ende des Zweiten Weltkrieges. Ziel ist es, dieses Archiv digital zugänglich zu machen, die beteiligten Personen sowie deren Netzwerke und dessen geographische Ausdehnung sichtbar zu machen.





%--------------------------------------------------------------------------------------%
%-----------------------------      T A B L E        ----------------------------------%
%-----------------------------        O F            ----------------------------------%
%-----------------------------   C O N T E N T S     ----------------------------------%
%--------------------------------------------------------------------------------------%



\renewcommand*\contentsname{Table of Contents} % This controls the title of your table of contents.
{
\hypersetup{linkcolor=}
\setcounter{tocdepth}{5} % Sets the maximum sublevel to be displayed within the table of contents.
\tableofcontents
}
\newpage
\pagenumbering{arabic}\setstretch{1.5} % Overwrites the previous command, pages are counted as normal from this point.


%--------------------------------------------------------------------------------------%
%--------------------------------------------------------------------------------------%
%------------------------      I N T R O D U C T I O N     ----------------------------%
%--------------------------------------------------------------------------------------%
%--------------------------------------------------------------------------------------%

\section{Introduction}

What is LaTeX? You may have not heard of it before, but you have almost certainly seen it used somewhere, even without knowing. LaTeX is, in short, a language that describes both content \textbf{and} layout of a text, as opposed to regular word processors like \emph{Microsoft Word}. It is a very powerful tool for scientific publication and, when used correctly, gives you almost infinite control over the layout of your document and makes it look very professional.

\textbf{Note}: This is \textbf{not} a guide on how to write a paper / thesis. If you need guidance on those topics, check the appropriate DH
Lab \href{https://dhlab.philhist.unibas.ch/en/master-course/seminar-masterarbeit-masterpruefung/}{resources} or ask a supervisor.

Instead, this template is meant to introduce you to what LaTeX can do and give you a starting point to use it on your own. While it is of course possible to install LaTeX locally and use it this way, it is recommended to use it via \href{https://www.overleaf.com}{Overleaf}, an online LaTeX editor that requires no installation and runs completely in the browser. This has multiple benefits over a local install: It works regardless of your device, your document and its associated files are always up to date and secure. All in all, it is clearly the most convenient and easy way to use LaTeX. This of course requires an internet connection to be used, which might be a downside to you.

If you have questions that are not addressed within this template, Overleaf offers \href{https://www.overleaf.com/learn/latex/Tutorials}{tutorials} on nearly everything as well as a comprehensive \href{https://www.overleaf.com/learn/latex/Learn_LaTeX_in_30_minutes}{guide} to understand LaTeX in 30 minutes. If all else fails, you can also ask ChatGPT. From experience, it is very adept at writing and correcting LaTeX-code. As you can already see in this chapter, you can put clickable links in your document.


%--------------------------------------------------------------------------------------%
%--------------------------------------------------------------------------------------%
%------------------------          S E C T I O N 1         ----------------------------%
%--------------------------------------------------------------------------------------%
%--------------------------------------------------------------------------------------%
\section{Structure of a document}

The most important file when using LaTeX is your \texttt{.tex} file. This is the file in which you write the text for your paper / thesis, while also describing its layout.

Each file starts with a preamble, in which you can change metadata and other settings of the document and also import and set up packages you want to use. In this template, the preamble is deliberately kept as short and concise as possible, but can of course be extended by you. Images, links, references, code etc. can be embedded within the document and will be rendered when the document is compiled. Pages are auto-numbered, starting after the table of contents.

The basic structure of a \texttt{.tex} file looks like this:

\begin{verbatim}
-----

This is the document preamble
\usepackage{example}

-----

\begin{document}

This is my Master's Thesis.

\end{document}
\end{verbatim}

Segmenting your document into sections and subsections works like this:


%--------------------------------------------------------------------------------------%
%--------------------------------------------------------------------------------------%
%------------------------          S E C T I O N 2         ----------------------------%
%--------------------------------------------------------------------------------------%
%--------------------------------------------------------------------------------------%
\section{This is a section}

\subsection{This is a sub-section}

\subsubsection{This is a sub-sub-section}

\paragraph{This is a paragraph}

\subparagraph{This is a sub-paragraph}


Up to 5 sublevels are possible, as you can see. By default, the sections are automatically numbered while the (sub)-paragraphs are not. This of course can be changed by you. The table of contents will be automatically generated (including said numbering and page numbers) upon compiling your document. \textbf{You do not need to write it yourself.}

See \href{https://www.overleaf.com/learn/latex/Sections_and_chapters}{this} part of the tutorial for more information about sections and chapters.


%--------------------------------------------------------------------------------------%
%--------------------------------------------------------------------------------------%
%------------------------          S E C T I O N 3         ----------------------------%
%--------------------------------------------------------------------------------------%
%--------------------------------------------------------------------------------------%

\section{Formatting your document}

LaTeX supports a variety of formatting options:

\emph{You can write in italics.}

\textbf{Or bold.}

\textbf{\emph{Or both at the same time.}}

You can also superscript\textsuperscript{text} or subscript\textsubscript{text}.

You can also write lists (unordered or ordered):

\begin{itemize}
\tightlist
\item
    Example list
\item 
    This one is unordered in the form of bullet points
\end{itemize}

\begin{enumerate}
\tightlist
\item
    This
\item
    is
\item
    an
\item
    ordered
\item
    list
\end{enumerate}

and use tables (with various formatting options, this is just an example):

\begin{table}[h!]
\centering
\begin{tabular}{|c c c|} 
 \hline
 Col1 & Col2 & Col3 \\ [0.5ex] 
 \hline\hline
 1 & 2 & 3 \\ 
 2 & 3 & 4 \\
 3 & 4 & 5 \\
 4 & 5 & 6 \\
 5 & 6 & 7 \\ [1ex] 
 \hline
\end{tabular}
\caption{Example table}
\end{table}

This is a more complex table, but it shows what is theoretically possible. As you can see, the tables are automatically numbered as well, as long as you give them a caption.

It is possible to automatically generate a list of tables at any point of the document. To do this, use the command \texttt{\textbackslash{listoftables}}:

\listoftables

You can also insert page breaks like this:

\newpage{}

These are only some examples, please see the respective sections in the tutorial for more: \href{https://www.overleaf.com/learn/latex/Bold%2C_italics_and_underlining}{text formatting}, \href{https://www.overleaf.com/learn/latex/Lists}{lists} and \href{https://www.overleaf.com/learn/latex/Tables}{tables}.


%--------------------------------------------------------------------------------------%
%--------------------------------------------------------------------------------------%
%------------------------          S E C T I O N 4
----------------------------%
%--------------------------------------------------------------------------------------%
%--------------------------------------------------------------------------------------%

\newpage{}
\printbibliography[
heading=bibintoc,
title={References} % title of the 'references' section, change this if necessary
]


\end{document}
