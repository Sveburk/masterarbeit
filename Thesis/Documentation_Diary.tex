\documentclass{article}
\usepackage[utf8]{inputenc}
\usepackage[ngerman]{babel}
\usepackage{pifont}  % Für die Kästchen und Häkchen-Symbole
\usepackage{minted}  % Für Python-Code
\usepackage{xcolor}  % Zum Definieren und Verwenden von Farben
\definecolor{LightGray}{gray}{0.9}
\usepackage[hidelinks]{hyperref} % Für die Hyperlinks im Inhaltsverzeichnis
\usepackage[a4paper, left=2.5cm, right=2.5cm, top=2cm, bottom=2cm]{geometry}

\title{\Huge Masterarbeit Tagebuch}
\author{Sven Burkhardt}
\date{}

\begin{document}
\maketitle
\noindent\hrulefill

\newpage % Neue Seite für das Inhaltsverzeichnis
\tableofcontents

%dieses Dokument wird blogartig geführt. Neuste Einträge sind oben einzufügen. 

%___________________________________________________________________________
% Vorlage 
\noindent\hrulefill
\section{Vorlage \small dd.mm.yyyy } % Kurztitel

\subsection*{Kurzbeschreibung}
Hier kurze Zusammenfassung einfügen.

\subsection*{Erledigte Aufgaben}
\subsubsection*{\small Aufgabentitel}
\ding{51} Aufgabenbeschreibung.\\
\textbf{Ergebnis}: Ergebnisbeschreibung.

\subsection*{Nächste Schritte}
\ding{113} Beschrieb der nächsten Aufgabe.

\subsection*{Offene Fragen}
\ding{113} Beschreibung der offenen Frage.

%___________________________________________________________________________
\newpage % Nach dem Inhaltsverzeichnis eine neue Seite
%___________________________________________________________________________
\noindent\hrulefill
\section{Tagging der JPGEGs im AppleFinder \small 26.10.24 } % Kurztitel

\subsection*{Kurzbeschreibung}

Überlegung:
JPEGs sollen bereits im Apple Finder mit Tags versehen werden, um eine effiziente, automatisierte Transkription der Chorunterlagen des Männerchors Murg zu ermöglichen. Geplant ist die Kombination von ChatGPT und Transkribus zur Erkennung unterschiedlicher Dokumententypen. Ein Tag-System, bestehend aus \textbf{„Maschinell“} für maschinengeschriebene und \textbf{„Handschrift“} für handschriftliche Dokumente, gewährleistet die gezielte Zuordnung zur jeweils geeigneten OCR-Software \textit{(Maschinenschrift mit ChatGPT, Handschrift mit Transkribus "German Giant")}.

Dokumente, die sowohl maschinell erstellten Text als auch handschriftliche Elemente enthalten, werden entsprechend ihrer Hauptinformationsgehalt getaggt. Zusätzlich erhalten alle Dokumente mit Unterschriften den Tag \textbf{„Unterschrift“}, um eine gezielte Verarbeitung dieser Elemente sicherzustellen.

\subsection{AppleFinder Tags}

\textcolor{yellow}{\Large $\bullet$} \textbf{Handschrift} \\
\textcolor{blue}{\Large $\bullet$} \textbf{Maschinell} \\
\textcolor{gray}{\Large $\bullet$} \textbf{mitUnterschrift} \\
\textcolor{purple}{\Large $\bullet$} \textbf{Bild} \\

\subsection*{Erledigte Aufgaben}
\subsubsection*{\small Handschriften tagging}
\ding{51} taggen \textcolor{yellow}{\Large $\bullet$} \textbf{Handschrift} in AppleFinder. \\
\ding{51} taggen \textcolor{blue}{\Large $\bullet$} \textbf{Maschinell} in AppleFinder. \\
\ding{51} taggen \textcolor{purple}{\Large $\bullet$} \textbf{Bild} \\
\ding{51} taggen \textcolor{gray}{\Large $\bullet$} \textbf{mitUnterschrift} \\
\\
\textbf{Ergebnis}: Handschriften, Maschinell, Bilder und alle handschriftlichen Unterschriften getaggt

\subsection*{Nächste Schritte}
\ding{113} Skripte schreiben, um maschinelle Text zu extrahieren 
\ding{113} Transkribus für Handschriftliches anschmeißen. \\
\ding{113} Nach Gemeinsamkeiten in den Texten suchen, um automatisierte Abfrage für ChatGPT zu erstellen. \\ 
\ding{113} Ggf. Aufteilung in unterschiedliche Korpora (Briefe handschr. Briefe schreibmaschine, Zeitungsunterlagen.)\\
\ding{113} Transkribus für Handschriften verwenden. \\

\subsection*{Offene Fragen}
\ding{113} Sollen die Bilder gelöscht werden?\\

%___________________________________________________________________________

\noindent\hrulefill
\section{JPG Datenbereinigung - leere Seiten löschen \small 25.10.2024 } % Kurztitel

\subsection*{JPG Datenbereinigung}
Alle JPGs ohne Inhalt, also beispielsweise Rückseiten, werden gelöscht. Regel: sobald etwas handschriftlich oder gedruckt auf einer Seite steht, bleibt es erhalten. Im Moment sind auch Bilder (Bsp. Postkarten inbegriffen.Bilder mit Taggs 
\subsection{Anmerkung}
Geschichte/Chronik/Gründung des Männerchors in Akte 323

\subsection*{Erledigte Aufgaben}
\subsubsection*{\small JPG Datenbereinigung }
\ding{51}Alle JPGs ohne Inhalt, also beispielsweise Rückseiten, werden gelöscht.\\
\textbf{Ergebnis}: Reiner JPG Korpus mit Schriftgut, aber auch Bildern (bspw. Postkarten)

\subsection*{Nächste Schritte}
\ding{113} Nach Gemeinsamkeiten in den Texten suchen, um automatisierte Abfrage für ChatGPT zu erstellen. \\ 
\ding{113}Ggf. Aufteilung in unterschiedliche Korpora (Briefe handschr. Briefe schreibmaschine, Zeitungsunterlagen.)\\
\ding{113} Transkribus für Handschriften verwenden.

\subsection*{Offene Fragen}
\ding{113} Sollen die Bilder gelöscht werden?\\
\ding{113} Handschriftliche, maschinengeschriebene und gemischte Daten taggen? Ggf. erst später mit ChatGPT.\\
\ding{113} Transkribus für Handschriften verwenden?

%___________________________________________________________________________

\noindent\hrulefill

\section{Datennormalisierung PDF zu JPEG \small 24.10.2024}
\subsection*{Kurzbeschrieb}
Heute zwei Python-Skripte zur Normalisierung der Akten geschrieben:

\subsection*{Erledigte Augaben}
\subsubsection*{\small PDF zu JPG Konvertierung}
\ding{51} Skript \textbf{\textit{JPEG-to-PDF.py}} geschrieben.  \\
\textbf{Ergebnis} Alle PDF-Seiten in JPGs umgewandelt, Dateinamen mit Seitenzahlen formatiert.

\subsubsection*{\small Prüfung der Aktennummern}
\ding{51} Skript \textbf{\textit{Check-if-all-files-complete.py}} geschrieben. \\
\textbf{Ergebnis}: Überprüft, ob Akten von 001 bis 425 vorhanden sind. Alle Akten sind vollständig in JPG umgewandelt.

\subsection*{Nächste Schritte}
\ding{113} Daten für OCR-Bereinigung vorbereiten, leere Seiten manuell entfernen.
\subsection*{Offene Fragen}
\ding{113} Handschriftliche, maschinengeschriebene und gemischte Daten taggen? Ggf. erst später mit ChatGPT.

\noindent\hrulefill

%___________________________________________________________________________
\end{document}

