%--------------------------------------------------------------------------------------%
%--------------------------------------------------------------------------------------%
%-----------------------------    S E T T I N G S     ---------------------------------%
%--------------------------------------------------------------------------------------%
%--------------------------------------------------------------------------------------%



% Options for packages loaded elsewhere
\PassOptionsToPackage{unicode}{hyperref}
\PassOptionsToPackage{hyphens}{url}
\PassOptionsToPackage{dvipsnames,svgnames,x11names}{xcolor}
%
\documentclass[12pt, a4paper, ngerman, bidi=default]{article}

%--------------------------------------------------------------------------------------%
%-----------------------------    S E T T I N G S     ---------------------------------%
%-----------------------------      P A K E T E        --------------------------------%
%--------------------------------------------------------------------------------------%
\usepackage[utf8]{inputenc}
\usepackage[T1]{fontenc}
\usepackage[fixed]{fontawesome5} %Fontawesome für Icons und Symbole; siehe https://mirrors.ibiblio.org/CTAN/fonts/fontawesome5/doc/fontawesome5.pdf
\usepackage{amsmath,amssymb}
\usepackage{xcolor}
\usepackage{tcolorbox}
\usepackage{afterpage}
\usepackage{hyperref}
\usepackage{graphicx}
\usepackage{subcaption}  % Für die Verwendung von subfigure
\usepackage{setspace}  % Für den Befehl \setstretch
\usepackage{transparent}
\usepackage{tikz}
\usepackage{eso-pic} % Für Hintergrundbilder
\usepackage{fvextra}     % Muss vor csquotes geladen werden
\usepackage{csquotes}    % Nach fvextra laden
\usepackage[authordate,backend=biber,language=ngerman]{biblatex-chicago}
\addbibresource{assets/Literature_Bib/literatur.bib}
\usepackage{minted} % Für Python-Code im Text

\usepackage[ngerman]{babel}
\usepackage{pifont}  % Für die Kästchen und Häkchen-Symbole
\usepackage{minted}  % Für Python-Code
\usepackage{xcolor}  % Zum Definieren und Verwenden von Farben
\definecolor{LightGray}{gray}{0.9}
\definecolor{UniRot}{HTML}{D20537}                  %Corperate Design Farben Uni Basel 
\definecolor{UniAnthrazit}{HTML}{46505A}            %Corperate Design Farben Uni Basel 
\definecolor{UniMint}{HTML}{A5D7D2}                 %Corperate Design Farben Uni Basel 
\usepackage{array}
\usepackage{colortbl}
\usepackage{booktabs}
\usepackage{ragged2e}
\hypersetup{
    colorlinks=true,
    linkcolor=lightblue
}
\usepackage{iftex}
\usepackage{csquotes}
\ifPDFTeX%
\usepackage{lmodern}  % Nur für PDFTeX
\else
    \usepackage{fontspec} % Für XeTeX und LuaTeX
\fi

% Use upquote if available, for straight quotes in verbatim environments
\IfFileExists{upquote.sty}{\usepackage{upquote}}{}

% Paragraph spacing configuration depending on the class
\makeatletter
\@ifundefined{KOMAClassName}{% if non-KOMA class
  \IfFileExists{parskip.sty}{%
    \usepackage{parskip}
  }{% else
    \setlength{\parindent}{0pt}
    \setlength{\parskip}{0pt}}
}{% if KOMA class
  \KOMAoptions{parskip=half}}
\makeatother

\usepackage[x11names,table]{xcolor}
\usepackage[lmargin=2.5cm,rmargin=2.5cm,tmargin=2cm,bmargin=2cm]{geometry}
\setlength{\emergencystretch}{3em} % prevent overfull lines
\setcounter{secnumdepth}{-\maxdimen} % remove section numbering
% Make \paragraph and \subparagraph free-standing
\makeatletter
\ifx\paragraph\undefined\else
  \let\oldparagraph\paragraph%
  \renewcommand{\paragraph}{
    \@ifstar%
      \xxxParagraphStar%
      \xxxParagraphNoStar%
  }
  \newcommand{\xxxParagraphStar}[1]{\oldparagraph*{#1}\mbox{}}
  \newcommand{\xxxParagraphNoStar}[1]{\oldparagraph{#1}\mbox{}}
\fi
\ifx\subparagraph\undefined\else
  \let\oldsubparagraph\subparagraph%
  \renewcommand{\subparagraph}{
    \@ifstar%
      \xxxSubParagraphStar%
      \xxxSubParagraphNoStar%
  }
  \newcommand{\xxxSubParagraphStar}[1]{\oldsubparagraph*{#1}\mbox{}}
  \newcommand{\xxxSubParagraphNoStar}[1]{\oldsubparagraph{#1}\mbox{}}
\fi
\makeatother


\providecommand{\tightlist}{%
  \setlength{\itemsep}{0pt}\setlength{\parskip}{0pt}}\usepackage{longtable,booktabs,array}
\usepackage{calc} % for calculating minipage widths
% Correct order of tables after \paragraph or \subparagraph
\usepackage{etoolbox}
\makeatletter
\patchcmd\longtable{\par}{\if@noskipsec\mbox{}\fi\par}{}{}
\makeatother
% Allow footnotes in longtable head/foot
\IfFileExists{footnotehyper.sty}{\usepackage{footnotehyper}}{\usepackage{footnote}}
\makesavenoteenv{longtable}
\usepackage{graphicx}
\makeatletter
\def\maxwidth{\ifdim\Gin@nat@width>\linewidth\linewidth\else\Gin@nat@width\fi}
\def\maxheight{\ifdim\Gin@nat@height>\textheight\textheight\else\Gin@nat@height\fi}
\makeatother
% Scale images if necessary, so that they will not overflow the page
% margins by default, and it is still possible to overwrite the defaults
% using explicit options in \includegraphics[width, height, ...]{}
\setkeys{Gin}{width=\maxwidth,height=\maxheight,keepaspectratio}
% Set default figure placement to htbp
\makeatletter
\def\fps@figure{htbp}
\makeatother
% definitions for citeproc citations
\NewDocumentCommand\citeproctext{}{}
\NewDocumentCommand\citeproc{mm}{%
  \begingroup\def\citeproctext{#2}\cite{#1}\endgroup}
\makeatletter
 % allow citations to break across lines
 \let\@cite@ofmt\@firstofone%
 % avoid brackets around text for \cite:
 \def\@biblabel#1{}
 \def\@cite#1#2{{#1\if@tempswa, #2\fi}}
\makeatother
\newlength{\cslhangindent}
\setlength{\cslhangindent}{1.5em}
\newlength{\csllabelwidth}
\setlength{\csllabelwidth}{3em}
\newenvironment{CSLReferences}[2] % #1 hanging-indent, #2 entry-spacing
 {\begin{list}{}{%
  \setlength{\itemindent}{0pt}
  \setlength{\leftmargin}{0pt}
  \setlength{\parsep}{0pt}
  % turn on hanging indent if param 1 is 1
  \ifodd #1 \else % chktex 1
   \setlength{\leftmargin}{\cslhangindent}
   \setlength{\itemindent}{-1\cslhangindent}
  \fi
  % set entry spacing
  \setlength{\itemsep}{#2\baselineskip}}}
 {\end{list}}
\usepackage{calc}
\newcommand{\CSLBlock}[1]{\hfill\break\parbox[t]{\linewidth}{\strut\ignorespaces#1\strut}}
\newcommand{\CSLLeftMargin}[1]{\parbox[t]{\csllabelwidth}{\strut#1\strut}}
\newcommand{\CSLRightInline}[1]{\parbox[t]{\linewidth-\csllabelwidth}{\strut#1\strut}} %chktex 8
\newcommand{\CSLIndent}[1]{\hspace{\cslhangindent}#1}

\ifpdf%
  \usepackage{authblk} % Paket für Affiliations
  \usepackage{orcidlink} % Paket für ORCID-Link
  \usepackage{pdfpages} % Paket zum Einbinden von PDFs
\makeatletter
\@ifpackageloaded{caption}{}{\usepackage{caption}}
\AtBeginDocument{%
\ifdefined\contentsname%
  \renewcommand*\contentsname{Inhaltsverzeichnis}
\else
  \newcommand\contentsname{Inhaltsverzeichnis}
\fi
\ifdefined\listfigurename%
  \renewcommand*\listfigurename{Abbildungsverzeichnis}
\else
  \newcommand\listfigurename{Abbildungsverzeichnis}
\fi
\ifdefined\listtablename%
  \renewcommand*\listtablename{Tabellenverzeichnis}
\else
  \newcommand\listtablename{Tabellenverzeichnis}
\fi
\ifdefined\figurename%
  \renewcommand*\figurename{Abbildung}
\else
  \newcommand\figurename{Abbildung}
\fi
\ifdefined\tablename%
  \renewcommand*\tablename{Tabelle}
\else
  \newcommand\tablename{Tabelle}
\fi
}
\@ifpackageloaded{float}{}{\usepackage{float}}
\floatstyle{ruled}
\@ifundefined{c@chapter}{\newfloat{codelisting}{h}{lop}}{\newfloat{codelisting}{h}{lop}[chapter]}
\floatname{codelisting}{Listing}
\providecommand{\listoflistings}{\listof{codelisting}{Listingverzeichnis}}
\makeatother
\makeatletter
\makeatother
\makeatletter
\@ifpackageloaded{caption}{}{\usepackage{caption}}
\@ifpackageloaded{subcaption}{}{\usepackage{subcaption}}
\makeatother

\babelprovide[main,import]{ngerman}
% get rid of language-specific shorthands (see #6817):
\let\LanguageShortHands\languageshorthands%
\def\languageshorthands#1{}
\usepackage{bookmark}

\IfFileExists{xurl.sty}{\usepackage{xurl}}{} % add URL line breaks if available
\urlstyle{same} % disable monospaced font for URLs
\hypersetup{
  pdftitle={Konzeption für AG Masterarbeit am
17.01.2025},
  pdfauthor={Sven Burkhardt},
  pdflang={de},
  colorlinks=true,
  linkcolor={blue},
  filecolor={Maroon},
  citecolor={Blue},
  urlcolor={Blue},
  pdfcreator={LaTeX via pandoc}
}

\usepackage{etoolbox}
\makeatletter
\providecommand{\subtitle}[1]{% add subtitle to \maketitle
  \apptocmd{\@title}{\par {\large #1 \par}}{}{}
}
\makeatother
\title{\vspace*{4cm} \LARGE Von Papier zur digitalen Netzwerkanalyse 
\color{UniMint} \rule{8cm}{0.4pt} \\  
\vspace{0.2cm}  
\color{white}\large Digitalisierung, Modellierung und Untersuchung\\historischer Vereinsakten\\mit Machine Learning und Nodegoat}
\color{UniMint} \rule{8cm}{0.4pt} \\  
\vspace{0.2cm}

\usepackage{etoolbox}
\makeatletter
\providecommand{\subtitle}[1]{% add subtitle to \maketitle
  \apptocmd{\@title}{\par {\large #1 \par}}{}{}
}
\makeatother
\subtitle{}
\author{Sven Burkhardt}
\date{2025-01-17} % chktex 8

%--------------------------------------------------------------------------------------%
%--------------------------------------------------------------------------------------%
%----------------------------- T I T E L B L A T T   ----------------------------------%
%--------------------------------------------------------------------------------------%
%--------------------------------------------------------------------------------------%
\begin{document}
\begin{titlepage}
    
% Setzt die Schriftfarbe auf Weiß
\color{white}
\pagecolor[HTML]{46505A } %Seitenfarbe in Uni Basel Anthrazit D20537 (rot)
\pagenumbering{gobble}    % Verhindert die Anzeige der Seitennummer auf dem Titelblatt
\date{}
\author{}
\maketitle
\begin{center}
  \author{\LARGE{\author{\vspace{-0.5cm}Sven Burkhardt}}}\\
  \vspace{4mm}
  \large{\orcidlink{0009-0001-4954-4426} {0009-0001-4954-4426}}\\ % chktex 8 % Orcid Link und Nummer
  \begin{figure}[h]
    \centering
    \color{white}
    \large{\href{https://dhlab.philhist.unibas.ch/en/persons/sven-burkhardt/}{{\hspace*{0.5mm}\includegraphics[height=4.5
  mm]{./assets/Logos/Uni_basel_logo_white.png}}\hspace{3.4mm}\color{white} 17-056-912}}\\ % chktex 8 %logo Unibas + Link + Immatrikulationsnummer
    
    \faIcon[regular]{calendar-alt}\date{\hspace*{2mm}17-01-2025} % chktex 8
  \end{figure}
  \setcounter{figure}{0}
\end{center}


% ------------ Hexagon grafik beginn -----------
\centering
\AddToShipoutPictureBG*{%
    \put(0,-40){%
        \includegraphics[width=\paperwidth]{./assets/Logos/Hexagon_Deko}
    }
}
\centering
\AddToShipoutPictureBG*{%
    \put(0,810){%
        \includegraphics[width=\paperwidth]{./assets/Logos/Hexagon_Deko}
    }
}
\centering
\AddToShipoutPictureBG*{%
    \put(33,752){%
        \includegraphics[width=\paperwidth]{./assets/Logos/Hexagon_Deko}
    }
}
\centering
\AddToShipoutPictureBG*{%
    \put(-99,752){%
        \includegraphics[width=\paperwidth]{./assets/Logos/Hexagon_Deko}
    }
}



\noindent % Verhindert Einzug des nachfolgenden Textes
% ------------ Hexagon grafik ende -----------



\begin{center}
    \vfill
    \begin{figure}
        \centering
        \begin{subfigure}{.3\textwidth}
          \centering
          \includegraphics[width=.8\linewidth]{./assets/Logos/uni-basel-logo-en_white.png}
        \end{subfigure}%
        \begin{subfigure}{.3\textwidth}
          \centering
          \includegraphics[width=.8\linewidth]{./assets/Logos/dhlab-logo-white.png}
        \end{subfigure}
        \end{figure}
        \setcounter{figure}{0}

    University of Basel\\
    Digital Humanities Lab\\
    Switzerland
\end{center}


\newpage
\newpage
\pagenumbering{arabic}
\color{black}           % Setzt die Schriftfarbe auf Schwarz für die folgenden Seiten
\setstretch{1.5}
\thispagestyle{empty}
\end{titlepage}
\newpage
%________________

%________________

%--------------------------------------------------------------------------------------%
%--------------------------------------------------------------------------------------%
%-----------------------------   A B S T R A C T     ----------------------------------%
%--------------------------------------------------------------------------------------%
%--------------------------------------------------------------------------------------%


\pagecolor{white}  
\color{black}  % Textfarbe zurücksetzen
\section*{Abstract}

Diese Arbeit befasst sich mit dem Archiv des Männerchor Murg in den Jahren der Weimarer Republik bis zum Ende des Zweiten Weltkrieges. Ziel ist es, dieses Archiv digital zugänglich zu machen, die beteiligten Personen sowie deren Netzwerke und dessen geographische Ausdehnung sichtbar zu machen.

\newpage

%--------------------------------------------------------------------------------------%
%-----------------------------      T A B L E        ----------------------------------%
%-----------------------------        O F            ----------------------------------%
%-----------------------------   C O N T E N T S     ----------------------------------%
%--------------------------------------------------------------------------------------%



\renewcommand*\contentsname{Inhaltsverzeichnis} % This controls the title of your table of contents.
{
\hypersetup{linkcolor=}
\setcounter{tocdepth}{5} % Sets the maximum sublevel to be displayed within the table of contents.
\tableofcontents
}
\newpage
\pagenumbering{arabic}\setstretch{1.5} % Overwrites the previous command, pages are counted as normal from this point.


%--------------------------------------------------------------------------------------%
%--------------------------------------------------------------------------------------%
%------------------------      I N T R O D U C T I O N     ----------------------------%
%--------------------------------------------------------------------------------------%
%--------------------------------------------------------------------------------------%

\section{Einleitung}
\subsection{Ziel und Relevanz der Arbeit}
\subsection{Forschungsstand und Forschungslücke}
\subsection{Formulierung der Forschungsfrage}
\subsection{Aufbau der Arbeit}

\newpage
%--------------------------------------------------------------------------------------%
%--------------------------------------------------------------------------------------%
%------------------------          Historischer Kontext        ----------------------------%
%--------------------------------------------------------------------------------------%
%--------------------------------------------------------------------------------------%
\section{Historischer Kontext}
\subsection{Historische Einordnung des Zeitraums}
\subsection{Historische Einordnung des Vereins}
  %\subsubsection{Der Männerchor Murg im Umfeld der Weimarer Republik}
  %\subsubsection{Auswirkungen der NS-Diktatur auf den Verein}
  \subsubsection{Der Männerchor während des Zweiten Weltkriegs}
  \subsubsection{Politische Entwicklungen und ihre Auswirkungen auf das Vereinsleben}

  \newpage
%--------------------------------------------------------------------------------------%
%--------------------------------------------------------------------------------------%
%------------------------          Quellenbeschreibung        ----------------------------%
%--------------------------------------------------------------------------------------%
%--------------------------------------------------------------------------------------% 
\section{Quellenbeschreibung und Korpusaufbau}
  \subsection{Beschreibung des Archivbestands}
    \newpage

    \href{https://free.iiifhosting.com/iiif/959173f8d808ab12ad7847917f79e0e4bc974ebce0040a07afd4b8be3f10c234/}{Feldpost Beispiel}
%--------------------------------------------------------------------------------------%
%--------------------------------------------------------------------------------------%
%------------------------          Methodik        ----------------------------%
%--------------------------------------------------------------------------------------%
%--------------------------------------------------------------------------------------% 

    
\section{Methodischer Zugang}

\subsection{Digitale Erfassung und Strukturierung der Quellen}
    \subsubsection{Gliederung in Akten}
    \subsubsection{Digitalisierung und Transkription}
    \subsection{Digitalisierungsprozess und Herausforderungen}
    Hier gehört dringend dazu, dass die Quellen über einen längeren Zeitraum digitalisiert wurden. Das bedeutet, dass sich die Kameras geändert haben. Verwendet wurden primär ein IPad Pro 2nd Generation (2017) und ein IPad Air 4th Generation (2022). Die Verwendete Software ist die Scan-Funktion von Apple ICloud. Die Auswahl der Software war aus rein ökonomischen Gründen. Da das Digitalisierungsprojekt bereits 2018 begonen wurde, fehlten weitestgehend Grundlagenkenntnisse, die im Digital Humanities Studium vermittelt wurden. Berücksichtigt wurden jedoch einige Richtlinien, wie sie in den Archiv-Kursen des Bachelor-Geschichtsstudiums vermittelt wurden (gleichbleibende Beleuchtung, Hintergrund). Die Scanqualität ist daher oft nicht optimal, was zu problemen bei der OCR Erkennung mit OCR Software (Apple OCR, Adobe, etc.) führte. Aus diesem Grund wurden 75 Akten zunächst mit dem Model ”The German Giant I” mit einer CER von 8,30\% transkribiert. In insgesammt  mit insgesammt  4 Iterationen wurde eine Groundtruth für ein eigenes Modell erstellt, und gleichzeitig Personen, Orte, Daten und Organisationen getaggt. Hierzu wurde auch manuell OpenAIs CHatGPT 4o Modell verwendet, das für die Rechtschreibprüfung verwendet wurde. Tauchte ein Rechtschreibfehler im Text auf, wurde dieser manuell überprüft. War der Fehler bereits im Ursprungstext, so wurde der Tag ”sic” verwendet,und eine Korrektur beigefügt.\\
    Die so erstellten 70 Akten ergaben 158 Seiten zu insgesammt 22.155 Wörtern Groundtruth, womit dann ein eigenes Transkribus Modell (\href{https://app.transkribus.org/models/public/287793}{ModelID: 287793}) erstellt wurde. Es erreichte eine Accuracy (CER) von 6,58\%. Später wurden die verbleibenden 80 Akten nur noch mit diesem Modell transkribiert.
      \subsection{Wechsel von Linked Open Data (LOD zu Nodegoat)}
        \subsubsection{Definition und Nutzen von LOD}
        \subsubsection{Aufbau der LOD Ontologie}
        \subsubsection{Gründe für den Wechsel zu Nodegoat}
        \subsubsection{Nodegoat Modelierung}


  \subsection{Netzwerkanalyse als Methode}
        \subsubsection{Theoretischer Hintergrund der Netzwerkanalyse}
        \subsubsection{Ziele der Netzwerkanalyse im Kontext der Quellen}
        \subsubsection{Technische Umsetzung (Tools, Datenbankstruktur)}

\subsection{Normalisierung der Dateien --- von PDF zu JPEG}
\begin{minted}[
frame=lines,
framesep=2mm,
baselinestretch=1.2,
bgcolor=LightGray,
fontsize=\footnotesize,
linenos,
breaklines=true,   % Automatischer Zeilenumbruch
breakanywhere=true % Minted darf überall umbrechen (optional)
]{python}
import os
import fitz  # PyMuPDF

def convert_pdf_to_jpg(src_folder, dest_folder):
    # Überprüfen, ob der Zielordner existiert, und ihn ggf. erstellen
    if not os.path.exists(dest_folder):
        os.makedirs(dest_folder)

    # Durchgehen durch alle Dateien im Quellordner
    for root, dirs, files in os.walk(src_folder):
        for file in files:
            # Überprüfen, ob die Datei eine PDF-Datei ist
            if file.lower().endswith(".pdf"):
                # Vollständigen Pfad zur PDF-Datei erstellen
                pdf_path = os.path.join(root, file)
                # PDF-Datei öffnen
                doc = fitz.open(pdf_path)
                # Durch alle Seiten der PDF-Datei gehen
                for page_num in range(len(doc)):
                    page = doc[page_num]
                    # Seite in ein PixMap-Objekt umwandeln (für die Konvertierung in JPG)
                    pix = page.get_pixmap()
                    # Dateinamen ohne Dateiendung extrahieren
                    filename_without_extension = os.path.splitext(file)[0]
                    # Ausgabedateinamen erstellen mit führenden Nullen für die 
                    # Seitennummer
                    output_filename = f"{filename_without_extension}_S
                    {page_num + 1:03d}.jpg"

                    # Vollständigen Pfad zur Ausgabedatei erstellen
                    output_path = os.path.join(dest_folder, output_filename)
                    # Bild speichern
                    pix.save(output_path)
                # PDF-Datei schließen
                doc.close()
                
                # Erfolgsmeldung ausgeben
                print(f"{file} wurde erfolgreich umgewandelt und gespeichert
                in {dest_folder}")

# Pfade zu den Ordnern mit den PDF-Dateien (Quelle) und den JPG-Dateien (Ziel)
src_folder = r"/Users/svenburkhardt/Documents/D_Murger_Männer_Chor_Forschung/Scan_Männerchor/Männerchor_Akten_1925–1945/Scan_Männerchor_PDF"
dest_folder = r"/Users/svenburkhardt/Documents/D_Murger_Männer_Chor_Forschung/Masterarbeit/JPEG_Akten_Scans"


# Funktion aufrufen, um die Konvertierung durchzuführen
convert_pdf_to_jpg(src_folder, dest_folder)

\end{minted}
\newpage
%--------------------------------------------------------------------------------------%
%--------------------------------------------------------------------------------------%
%------------------------          Aufbau der Datenbank        ----------------------------%
%--------------------------------------------------------------------------------------%
%--------------------------------------------------------------------------------------%

\section{Aufbau der Datenbank}
    \subsection{Konzeption der Datenmodelieung}
      \subsubsection{Eigene Ontologie im Vergleich zu bestehenden Standards}
      \subsubsection{Verknüpfung von Personen, Orten und Ereignissen}
    
    \subsection{Implementierung der Datenbank}
      \subsubsection{Datenbankdesign}
      \subsubsection{Herausforderungen bei der Datenaufnahme}
      \subsubsection{Verknüpfung mit externen Quellen (z.B. Wikidata)}

    \newpage
%--------------------------------------------------------------------------------------%
%--------------------------------------------------------------------------------------%
%------------------------          Analyse der Netzwerke         ----------------------------%
%--------------------------------------------------------------------------------------%
%--------------------------------------------------------------------------------------%
\section{Analyse der Netzwerke }
  \subsection{Soziale Netzwerke des Vereinslebens}
    \subsubsection{Verbindungen zwischen Mitgliedern}
    \subsubsection{ Kooperationen mit anderen Vereinen}
\subsection{ Politische Netzwerke und deren Veränderungen}
    \subsubsection{Einfluss der NS-Diktatur auf die Netzwerke}
    \subsubsection{Feldpostkarten als Quelle für militärische Netzwerke}
    
 \subsection{ Geografische Ausdehnung der Netzwerke}
  \subsubsection{Einsatzorte der Chormitglieder während des Krieges}
  \subsubsection{ Lokale und überregionale Verbindungen}
  \newpage
%--------------------------------------------------------------------------------------%
%--------------------------------------------------------------------------------------%
%------------------------          Diskussion der Erebnisse         ----------------------------%
%--------------------------------------------------------------------------------------%
%--------------------------------------------------------------------------------------%
\section{Diskussion der Ergebnisse}
  \subsection{Sichtbarmachung der Netzwerke durch Nodegoat und Netzwerkanalyse}
  \subsection{Gibt es Veränderungen der Netzwerke im historischen Kontext?}
  \newpage
%--------------------------------------------------------------------------------------%
%--------------------------------------------------------------------------------------%
%------------------------          Fazit und Ausblick        ----------------------------%
%--------------------------------------------------------------------------------------%
%--------------------------------------------------------------------------------------%
\section{Fazit und Ausblick}
\subsection{Zusammenfassung der zentralen Erkenntnisse}
\subsection{Methodische Herausforderungen und Lösungen}
\subsection{Ausblick auf zukünftige Forschung und mögliche Erweiterungen der Datenbank}
\newpage
%--------------------------------------------------------------------------------------%
%--------------------------------------------------------------------------------------%
%------------------------           Bibliographie      --------------------------------%
%--------------------------------------------------------------------------------------%
%--------------------------------------------------------------------------------------%
\section{Bibliographie}


\newpage{}
%\printbibliography[
%heading=bibintoc,
%title={References} % title of the 'references' section, change this if necessary
%]

\end{document}
