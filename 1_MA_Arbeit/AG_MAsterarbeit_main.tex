%--------------------------------------------------------------------------------------%
%--------------------------------------------------------------------------------------%
%-----------------------------    S E T T I N G S     ---------------------------------%
%--------------------------------------------------------------------------------------%
%--------------------------------------------------------------------------------------%



% Options for packages loaded elsewhere
\PassOptionsToPackage{unicode}{hyperref}
\PassOptionsToPackage{hyphens}{url}
\PassOptionsToPackage{dvipsnames,svgnames,x11names}{xcolor}
%
\documentclass[12pt, a4paper, ngerman, bidi=default]{article}

%--------------------------------------------------------------------------------------%
%-----------------------------    S E T T I N G S     ---------------------------------%
%-----------------------------      P A K E T E        --------------------------------%
%--------------------------------------------------------------------------------------%
\usepackage[utf8]{inputenc}
\usepackage[T1]{fontenc}
\usepackage[fixed]{fontawesome5} %Fontawesome für Icons und Symbole; siehe https://mirrors.ibiblio.org/CTAN/fonts/fontawesome5/doc/fontawesome5.pdf
\usepackage{amsmath,amssymb}
\usepackage{xcolor}
\usepackage{tcolorbox}
\usepackage{afterpage}
\usepackage{hyperref}
\usepackage{graphicx}
\usepackage{subcaption}  % Für die Verwendung von subfigure
\usepackage{setspace}  % Für den Befehl \setstretch
\usepackage{transparent}
\usepackage{tikz}
\usepackage{eso-pic} % Für Hintergrundbilder
\usepackage{fvextra}     % Muss vor csquotes geladen werden
\usepackage{csquotes}    % Nach fvextra laden
\usepackage[authordate,backend=biber,language=ngerman]{biblatex-chicago}
%\addbibresource{assets/Literature_Bib/literatur.bib}                     %LITERATURVERZEICHNIS EINFÜGEN
\usepackage{minted}

\usepackage[ngerman]{babel}
\usepackage{pifont}  % Für die Kästchen und Häkchen-Symbole
\usepackage{minted}  % Für Python-Code
\usepackage{xcolor}  % Zum Definieren und Verwenden von Farben
\definecolor{LightGray}{gray}{0.9}
\definecolor{UniRot}{HTML}{D20537}                  %Corperate Design Farben Uni Basel 
\definecolor{UniAnthrazit}{HTML}{46505A}            %Corperate Design Farben Uni Basel 
\definecolor{UniMint}{HTML}{A5D7D2}                 %Corperate Design Farben Uni Basel 
\usepackage{array}
\usepackage{colortbl}
\usepackage{booktabs}
\usepackage{ragged2e} 
\hypersetup{
    colorlinks=true,
    linkcolor=lightblue
}
\usepackage{csquotes}
\usepackage{lmodern}
\ifPDFTeX\fi
\ifPDFTeX\else
    % xetex/luatex font selection
\fi
% Use upquote if available, for straight quotes in verbatim environments
\IfFileExists{upquote.sty}{\usepackage{upquote}}{}

% Paragraph spacing configuration depending on the class
\makeatletter
\@ifundefined{KOMAClassName}{% if non-KOMA class
  \IfFileExists{parskip.sty}{%
    \usepackage{parskip}
  }{% else
    \setlength{\parindent}{0pt}
    \setlength{\parskip}{0pt}}
}{% if KOMA class
  \KOMAoptions{parskip=half}}
\makeatother
\usepackage[x11names,table]{xcolor}
\usepackage[lmargin=2.5cm,rmargin=2.5cm,tmargin=2cm,bmargin=2cm]{geometry}
\setlength{\emergencystretch}{3em} % prevent overfull lines
\setcounter{secnumdepth}{-\maxdimen} % remove section numbering
% Make \paragraph and \subparagraph free-standing
\makeatletter
\ifx\paragraph\undefined\else
  \let\oldparagraph\paragraph
  \renewcommand{\paragraph}{
    \@ifstar
      \xxxParagraphStar
      \xxxParagraphNoStar
  }
  \newcommand{\xxxParagraphStar}[1]{\oldparagraph*{#1}\mbox{}}
  \newcommand{\xxxParagraphNoStar}[1]{\oldparagraph{#1}\mbox{}}
\fi
\ifx\subparagraph\undefined\else
  \let\oldsubparagraph\subparagraph
  \renewcommand{\subparagraph}{
    \@ifstar
      \xxxSubParagraphStar
      \xxxSubParagraphNoStar
  }
  \newcommand{\xxxSubParagraphStar}[1]{\oldsubparagraph*{#1}\mbox{}}
  \newcommand{\xxxSubParagraphNoStar}[1]{\oldsubparagraph{#1}\mbox{}}
\fi
\makeatother


\providecommand{\tightlist}{%
  \setlength{\itemsep}{0pt}\setlength{\parskip}{0pt}}\usepackage{longtable,booktabs,array}
\usepackage{calc} % for calculating minipage widths
% Correct order of tables after \paragraph or \subparagraph
\usepackage{etoolbox}
\makeatletter
\patchcmd\longtable{\par}{\if@noskipsec\mbox{}\fi\par}{}{}
\makeatother
% Allow footnotes in longtable head/foot
\IfFileExists{footnotehyper.sty}{\usepackage{footnotehyper}}{\usepackage{footnote}}
\makesavenoteenv{longtable}
\usepackage{graphicx}
\makeatletter
\def\maxwidth{\ifdim\Gin@nat@width>\linewidth\linewidth\else\Gin@nat@width\fi}
\def\maxheight{\ifdim\Gin@nat@height>\textheight\textheight\else\Gin@nat@height\fi}
\makeatother
% Scale images if necessary, so that they will not overflow the page
% margins by default, and it is still possible to overwrite the defaults
% using explicit options in \includegraphics[width, height, ...]{}
\setkeys{Gin}{width=\maxwidth,height=\maxheight,keepaspectratio}
% Set default figure placement to htbp
\makeatletter
\def\fps@figure{htbp}
\makeatother
% definitions for citeproc citations
\NewDocumentCommand\citeproctext{}{}
\NewDocumentCommand\citeproc{mm}{%
  \begingroup\def\citeproctext{#2}\cite{#1}\endgroup}
\makeatletter
 % allow citations to break across lines
 \let\@cite@ofmt\@firstofone
 % avoid brackets around text for \cite:
 \def\@biblabel#1{}
 \def\@cite#1#2{{#1\if@tempswa , #2\fi}}
\makeatother
\newlength{\cslhangindent}
\setlength{\cslhangindent}{1.5em}
\newlength{\csllabelwidth}
\setlength{\csllabelwidth}{3em}
\newenvironment{CSLReferences}[2] % #1 hanging-indent, #2 entry-spacing
 {\begin{list}{}{%
  \setlength{\itemindent}{0pt}
  \setlength{\leftmargin}{0pt}
  \setlength{\parsep}{0pt}
  % turn on hanging indent if param 1 is 1
  \ifodd #1
   \setlength{\leftmargin}{\cslhangindent}
   \setlength{\itemindent}{-1\cslhangindent}
  \fi
  % set entry spacing
  \setlength{\itemsep}{#2\baselineskip}}}
 {\end{list}}
\usepackage{calc}
\newcommand{\CSLBlock}[1]{\hfill\break\parbox[t]{\linewidth}{\strut\ignorespaces#1\strut}}
\newcommand{\CSLLeftMargin}[1]{\parbox[t]{\csllabelwidth}{\strut#1\strut}}
\newcommand{\CSLRightInline}[1]{\parbox[t]{\linewidth - \csllabelwidth}{\strut#1\strut}}
\newcommand{\CSLIndent}[1]{\hspace{\cslhangindent}#1}

\ifpdf
  \usepackage{authblk} % Paket für Affiliations
  \usepackage{orcidlink} % Paket für ORCID-Link
  \usepackage{pdfpages} % Paket zum Einbinden von PDFs
\makeatletter
\@ifpackageloaded{caption}{}{\usepackage{caption}}
\AtBeginDocument{%
\ifdefined\contentsname
  \renewcommand*\contentsname{Inhaltsverzeichnis}
\else
  \newcommand\contentsname{Inhaltsverzeichnis}
\fi
\ifdefined\listfigurename
  \renewcommand*\listfigurename{Abbildungsverzeichnis}
\else
  \newcommand\listfigurename{Abbildungsverzeichnis}
\fi
\ifdefined\listtablename
  \renewcommand*\listtablename{Tabellenverzeichnis}
\else
  \newcommand\listtablename{Tabellenverzeichnis}
\fi
\ifdefined\figurename
  \renewcommand*\figurename{Abbildung}
\else
  \newcommand\figurename{Abbildung}
\fi
\ifdefined\tablename
  \renewcommand*\tablename{Tabelle}
\else
  \newcommand\tablename{Tabelle}
\fi
}
\@ifpackageloaded{float}{}{\usepackage{float}}
\floatstyle{ruled}
\@ifundefined{c@chapter}{\newfloat{codelisting}{h}{lop}}{\newfloat{codelisting}{h}{lop}[chapter]}
\floatname{codelisting}{Listing}
\providecommand{\listoflistings}{\listof{codelisting}{Listingverzeichnis}}
\makeatother
\makeatletter
\makeatother
\makeatletter
\@ifpackageloaded{caption}{}{\usepackage{caption}}
\@ifpackageloaded{subcaption}{}{\usepackage{subcaption}}
\makeatother

\babelprovide[main,import]{ngerman}
% get rid of language-specific shorthands (see #6817):
\let\LanguageShortHands\languageshorthands
\def\languageshorthands#1{}
\usepackage{bookmark}

\IfFileExists{xurl.sty}{\usepackage{xurl}}{} % add URL line breaks if available
\urlstyle{same} % disable monospaced font for URLs
\hypersetup{
  pdftitle={Konzeption für AG Masterarbeit am
17.01.2025},
  pdfauthor={Sven Burkhardt},
  pdflang={de},
  colorlinks=true,
  linkcolor={blue},
  filecolor={Maroon},
  citecolor={Blue},
  urlcolor={Blue},
  pdfcreator={LaTeX via pandoc}
}

\usepackage{etoolbox}
\makeatletter
\providecommand{\subtitle}[1]{% add subtitle to \maketitle
  \apptocmd{\@title}{\par {\large #1 \par}}{}{}
}
\makeatother
\title{\vspace*{4cm}\Huge{Vorbereitung Treffen AG Masterarbeit}}
\subtitle {\large 74906-01 – Masterarbeiten in Digital Humanities}
\author{Sven Burkhardt}
\date{2025-01-17}

%--------------------------------------------------------------------------------------%
%--------------------------------------------------------------------------------------%
%----------------------------- T I T E L B L A T T   ----------------------------------%
%--------------------------------------------------------------------------------------%
%--------------------------------------------------------------------------------------%
\begin{document}
\begin{titlepage}
    
% Setzt die Schriftfarbe auf Weiß
\color{white}
\pagecolor[HTML]{46505A } %Seitenfarbe in Uni Basel Anthrazit D20537 (rot)
\pagenumbering{gobble}    % Verhindert die Anzeige der Seitennummer auf dem Titelblatt
\date{}
\author{}
\maketitle
\begin{center}
  \author{\LARGE{\author{\vspace{-0.5cm}Sven Burkhardt}}}\\
  \vspace{4mm}
  \large{\orcidlink{0009-0001-4954-4426} {0009-0001-4954-4426}}\\ % Orcid Link und Nummer
  \begin{figure}[h]
    \centering
    \color{white}
    \large{\href{https://dhlab.philhist.unibas.ch/en/persons/sven-burkhardt/}{{\hspace*{0.5mm}\includegraphics[height=4.5
  mm]{./assets/Logos/Uni_basel_logo_white.png}}\hspace{3.4mm}\color{white} 17-056-912}}\\ %logo Unibas + Link + Immatrikulationsnummer
    
    \faIcon[regular]{calendar-alt}\date{\hspace*{2mm}17-01-2025}
  \end{figure}
  \setcounter{figure}{0}
\end{center}


% ------------ Hexagon grafik beginn -----------
\centering
\AddToShipoutPictureBG*{%
    \put(0,-40){%
        \includegraphics[width=\paperwidth]{./assets/Logos/Hexagon_Deko}
    }
}
\centering
\AddToShipoutPictureBG*{%
    \put(0,810){%
        \includegraphics[width=\paperwidth]{./assets/Logos/Hexagon_Deko}
    }
}
\centering
\AddToShipoutPictureBG*{%
    \put(33,752){%
        \includegraphics[width=\paperwidth]{./assets/Logos/Hexagon_Deko}
    }
}
\centering
\AddToShipoutPictureBG*{%
    \put(-99,752){%
        \includegraphics[width=\paperwidth]{./assets/Logos/Hexagon_Deko}
    }
}



\noindent % Verhindert Einzug des nachfolgenden Textes
% ------------ Hexagon grafik ende -----------



\begin{center}
    \vfill
    \begin{figure}
        \centering
        \begin{subfigure}{.3\textwidth}
          \centering
          \includegraphics[width=.8\linewidth]{./assets/Logos/uni-basel-logo-en_white.png}
        \end{subfigure}%
        \begin{subfigure}{.3\textwidth}
          \centering
          \includegraphics[width=.8\linewidth]{./assets/Logos/dhlab-logo-white.png}
        \end{subfigure}
        \end{figure}
        \setcounter{figure}{0}

    University of Basel\\
    Digital Humanities Lab\\
    Switzerland
\end{center}


\newpage
\newpage
\pagenumbering{arabic}
\color{black}           % Setzt die Schriftfarbe auf Schwarz für die folgenden Seiten
\setstretch{1.5}
\thispagestyle{empty}
\end{titlepage}
%________________



%--------------------------------------------------------------------------------------%
%--------------------------------------------------------------------------------------%
%-----------------------------   A B S T R A C T     ----------------------------------%
%--------------------------------------------------------------------------------------%
%--------------------------------------------------------------------------------------%
\vfill
\newpage
\pagecolor{white} % Seitenfarbe Weiss
\justifying     % Aktiviert den Blocksatz
\begin{spacing}{1.5}
\section*{Abstract}

Report mit Auflistung von Forschungsfrage und Herangehensweise um Strategien und offene Fragen im Plenum der \textit{Arbeitsgemeinschaft Masterarbeit} zu besprechen. Diese Arbeit befasst sich mit dem Archiv des Männerchor Murg in den Jahren der Weimarer Republik bis zum Ende des Zweiten Weltkrieges. Ziel ist es, dieses Archiv digital zugänglich zu machen, die beteiligten Personen sowie deren Netzwerke und geographische Ausdehnung sichtbar zu machen.
%--------------------------------------------------------------------------------------%
\section{Aufgabenstellung}
\subsection {\textbf{Thema der Arbeit}}

Die Masterarbeit beschäftigt sich mit historischen Unterlagen zwischen 1925-1945 aus dem Archiv eines Männerchores. Diese habe ich bereits digitalisiert und in insgesamt 425 als Akten bezeichnete Sinnabschnitte gegliedert. Sie umfassen unterschiedliche Dokumententypen (Zeitungsartikel, (Feld-)Postkarten, Protokolle,und ähnliches. Ziel der Arbeit ist, diese Dokumente in eine Linked Open Data Datenbank (LOD) zu überführen, und damit eine Verknüpfung von Personen, Orten und Ereignissen innerhalb der Daten darstellen zu können. Der Vorteil, den LOD bietet ist es, weitere externe Online-Quellen zu verknüpfen, und so einzelne Dokumente in einen grösseren historischen Kontext eingliedern zu können. 
Der Schwerpunkt der Arbeit liegt auf der Anwendung von Netzwerkanalysen, um die Beziehungen zwischen den Akten, den darin vorkommenden Personen, sowie zeitgeschichtlichen Ereignissen zu visualisieren und durch diese Kontextualisierung besser zu verstehen. Dabei sollen insbesondere Bezüge zu lokalen und überregionalen historischen Entwicklungen untersucht werden.

\subsection{\textbf{Forschungsstand}} 
\textit{Kritische Übersicht über die bisherige Forschung zu eurem Thema.
Identifikation der Forschungslücke, die eure Arbeit schliessen soll.}

Zu den Quellen selbst gib es keinen Forschungsstand; sie wurden noch nie bearbeitet. Für die Netzwerkanalyse nutze ich das Übersichtswerk von Gamper et. al.\footcite{gamper_knoten_2015}. Für Linked Open Data soll das Buch \textit{"Linking Knowledge. Linked Open Data for Knowledge Organization and Visualization"}\footcite{richard_linking_2022} ebenso wie der Kursinhalt von Sepideh Alassi erste Aufschlüsse geben.

\subsection{\textbf{Fragestellung}}
\textit{Formulierung einer klaren und spezifischen Forschungsfrage.
Die Frage sollte wissenschaftlich relevant sein und einen Beitrag zum Verständnis des Themas leisten.}

 Wie können die historischen Netzwerke des Männerchors Murg durch die Anwendung von Linked Open Data und Netzwerkanalyse sichtbar gemacht werden, und welche Veränderungen lassen sich in diesen Netzwerken im Kontext der politischen Umwälzungen zwischen Weimarer Republik, NS-Diktatur und Zweitem Weltkrieg feststellen?

\subsection {\textbf{Methodisch-theoretischer Zugang}}
\textit{Beschreibung und Begründung der Methoden und Theorien, die für die Beantwortung der Fragestellung geeignet sind.}

 Die Methodik der Arbeit wird in zwei Aspekte unterteilt. An erster Stelle steht der \textbf{Aufbau der LOD Datenbank}. Hierfür wird eine umfassende Ontologie erstellt, die zu grossen Teilen auf bereits bestehende Ontologien bauen soll, um eine bessere Integration in andere Projekte ermöglichen zu können. Die Strukturierung ist in grossen Teilen bereits erfolgt, und kann im  \hyperref[sec:Anhang Datenbankstruktur]{Anhang Datenbankstruktur} eingesehen werden. Gleiches gilt für die zum aktuellen Zeitpunkt \hyperref[tab:bestehenden Ontologien]{bestehenden Ontologien}.\\
An zweiter Stelle soll eine quantitative Auswertung der Daten erfolgen, um den zeitlichen und geograhischen Verlauf des Netzwerks abbilden zu können.  

\subsection {\textbf{Beschreibung der Quellen/Daten}}
\textit{Falls bereits bekannt, Angaben zu den geplanten Quellen oder Daten.}

Das Archiv des Männerchors Murg umfasst Dokumente aus der Zeit zwischen der Weimarer Republik (ab 1925), der NS-Diktatur und dem Zweiten Weltkrieg. Der Bestand ist in 425 Akten gegliedert und enthält verschiedene Dokumenttypen wie Protokolle, Briefe, Feldpostkarten, Zeitungsartikel, Mitgliederlisten, Verwaltungsunterlagen und Veranstaltungsprogramme (vgl. \hyperref[sec:Anhang Datenbankstruktur]{Anhang} sehen). 

Zu den zentralen Dokumenten Briefe und Postkarten von Mitgliedern, die während des Zweiten Weltkriegs verschickt wurden sowie Protokolle, die Sitzungen und Entscheidungen des Vereins dokumentieren. Einige wenige Zeitungsartikel dokumentieren öffentliche Auftritte und Veranstaltungen des Chors, aber auch Auseinandersetzungen von Chormitgliedern mit der NSDAP. Politische Korrespondenzen zeugen jedoch von einer raschen und zunehmenden Gleichschaltung und Nazifizierung zumindest von Teilen des Vereins und der Mitglieder, andere äussern sich zwischen den Zeilen vorsichtig kritisch.
So zeigen auch Verwaltungsdokumente wie Mitgliederlisten und Abrechnungen Einblick in die Struktur und Organisation des Vereins - wer ist Parteimitglied, wer muss an die Front? Programme verdeutlichen das musikalische Repertoire des Chors -  und von wem er engagiert wird.
Der Korpus umfasst immer selektive Themen, die durch die Datenbank in Kontext gebracht werden sollen. Am wohl interessantesten sind dabei die wenigen Feldpostbriefe, die Rückschlüsse auf den Einsatzort der Chormitglieder bieten. Diese sind bereits im Rahmen eines Seminars recherchiert worden, und können mit grob orientierender Einordnung am  \href{https://storymaps.arcgis.com/stories/ec8a4b675cac476380df910304a47547}{Ende der Webseite} eingesehen werden. Dort finden sich auch Beispiele aus den Quellen. 

\subsection{\textbf{Entwurf einer Gliederung (optional)}}
\setstretch{1}
\begin{enumerate}
    \item \textbf{Einleitung}
    \begin{enumerate}
        \item \small Ziel und Relevanz der Arbeit
        \item \small Forschungsstand und Forschungslücke
        \item \small Formulierung der Forschungsfrage
        \item \small Aufbau der Arbeit
    \end{enumerate}
    
    \item \textbf{Historischer Kontext}
        \begin{enumerate}
        \item \small{ Historische Einordnung des Zeitraums}
        \item \small{ Historische Einordnung des Vereins}
            \begin{enumerate}
            \item \small Der Männerchor Murg im Umfeld der Weimarer Republik
            \item \small Auswirkungen der NS-Diktatur auf den Verein 
            \item \small Der Männerchor während des Zweiten Weltkriegs
            \item \small Politische Entwicklungen und ihre Auswirkungen auf das Vereinsleben
    \end{enumerate}
\end{enumerate}
    
    \item \textbf{Quellenbeschreibung und Korpusaufbau}
    \begin{enumerate}
        \item \small Beschreibung des Archivbestands
        \item \small Digitale Erfassung und Strukturierung der Quellen
        \begin{enumerate}
            \item \small Gliederung in Akten
            \item \small Herausforderungen bei der Digitalisierung (z.\,B.\ OCR-Erkennung)
        \end{enumerate}
    \end{enumerate}
    
    \item \textbf{Methodischer Zugang}
    \begin{enumerate}
        \item \small Einführung in Linked Open Data (LOD)
        \begin{enumerate}
            \item \small \textit{Definition und Nutzen von LOD}
        \end{enumerate}
        \item \textbf{Netzwerkanalyse als Methode}
        \begin{enumerate}
            \item \small{ \textit{Theoretischer Hintergrund der Netzwerkanalyse
            \item Ziele der Netzwerkanalyse im Kontext der Quellen
            \item Technische Umsetzung (Tools, Datenbankstruktur)}}
        \end{enumerate}
    \end{enumerate}
    
    \item \textbf{Aufbau der Datenbank}
    \begin{enumerate}
        \item \small { Konzeption der Ontologie}
        \begin{enumerate}
            \item \small {\textit{Eigene Ontologie im Vergleich zu bestehenden Standards
            \item Verknüpfung von Personen, Orten und Ereignissen}}
        \end{enumerate}
        \item Implementierung der Datenbank
        \begin{enumerate}
            \item \small {Datenbankdesign
            \item Herausforderungen bei der Datenaufnahme
            \item Verknüpfung mit externen Quellen (z.\,B.\ Wikidata)}
        \end{enumerate}
    \end{enumerate}
    
    \item \textbf{Analyse der Netzwerke }
    \begin{enumerate}
        \item \small{Soziale Netzwerke des Vereinslebens
        \begin{enumerate}
            \item \textit{Verbindungen zwischen Mitgliedern
            \item Kooperationen mit anderen Vereinen}
        \end{enumerate}
        \item Politische Netzwerke und deren Veränderungen
        \begin{enumerate}
            \item \textit{Einfluss der NS-Diktatur auf die Netzwerke
            \item Feldpostkarten als Quelle für militärische Netzwerke}
        \end{enumerate}
        \item Geografische Ausdehnung der Netzwerke
        \begin{enumerate}
            \item \textit{Einsatzorte der Chormitglieder während des Krieges
            \item Lokale und überregionale Verbindungen}
        \end{enumerate}}
    \end{enumerate}
    
    \item \textbf{ Diskussion der Ergebnisse}
    \begin{enumerate}
        \item Sichtbarmachung der Netzwerke durch LOD und Netzwerkanalyse
        \item Gibt es Veränderungen der Netzwerke im historischen Kontext?
    \end{enumerate}
    
    \item \textbf{Fazit und Ausblick}
    \begin{enumerate}
        \item Zusammenfassung der zentralen Erkenntnisse
        \item Methodische Herausforderungen und Lösungen
        \item Ausblick auf zukünftige Forschung und mögliche Erweiterungen der Datenbank
    \end{enumerate}

    \textbf{Literaturverzeichnis}
    
    \item \textbf{Anhang}
    
\end{enumerate}
\setstretch{1.5}

\newpage
\section{\textbf{Anhang}}
\subsection{Skizze der Datenbankstruktur}
\begin{tcolorbox}[colback=gray!20, colframe=gray!50, width=\dimexpr\textwidth-7cm\relax]
\textcolor{blue}{\ding{110}} Analogen Akten und damit \\
\hspace*{0.5cm}verbundene Klassifizierung.\\
\textcolor{yellow}{\ding{110}} Personen und deren Rollen.\\
\textcolor{orange}{\ding{110}} Organisationen.\\
\textcolor{green}{\ding{51}} grüne Haken sind bereits rudimentär \\\hspace*{0.5cm}in meiner Ontologie aufgebaut.
\end{tcolorbox}
\label{sec:Anhang Datenbankstruktur}
\vspace{1em}

\begin{figure}[htbp]
\centering
\includepdf[pages=-]{assets/Datenbank_Maennerchor_Struktur_FS2024_komp.pdf}
\end{figure}
\newpage
\subsubsection{Bestehenden Ontologien}
\label{tab:bestehenden Ontologien}
\begin{table}[h]
\centering
\renewcommand{\arraystretch}{1.3}  % Vergrößert den Zeilenabstand in der Tabelle
\begin{tabular}{@{}>{\ttfamily}l>{\raggedright\arraybackslash}p{5cm}>{\raggedright\arraybackslash}p{7cm}@{}}
\toprule
\rowcolor{gray!20} \multicolumn{3}{c}{\textbf{Interne Ontologien}} \\
\midrule
\textbf{Präfix} & \textbf{Ontologie/Quelle} & \textbf{Funktion} \\
\arrayrulecolor{lightgray}\midrule
: & Eigene Ontologie & Projektspezifische Klassen und Properties (bspw. Feldpostnummer)\\
\arrayrulecolor{black}\midrule
\rowcolor{gray!20} \multicolumn{3}{c}{\textbf{Externe Ontologien}} \\
\midrule
\textbf{Präfix} & \textbf{Ontologie/Quelle} & \textbf{Funktion} \\
\arrayrulecolor{lightgray}\midrule
schema: & Schema.org & Modellierung allgemeiner Konzepte (Person, Ort) \\
wd: & Wikidata & Verknüpfung mit externen Wikidata-Einträgen \\
owl: & Web Ontology Language & Komplexe Modellierung von Beziehungen \\
xsd: & W3C XML Schema & Datentypen (String, Datum) \\
rdfs: & RDF Schema & Strukturierung von Klassen und Eigenschaften \\
skos: & SKOS & Organisation kontrollierter Vokabulare, aktuell genutzt für alternative Label \\
\bottomrule
\end{tabular}
\caption{aktuell Verwendete Ontologien in der Männerchor-Datenbank (Stand: 13.01.25)}
\end{table}
\end{spacing}
\printbibliography
\end{document}