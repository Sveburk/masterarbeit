\documentclass{article}
\usepackage[utf8]{inputenc}
\usepackage[ngerman]{babel}
\usepackage{pifont}  % Für die Kästchen und Häkchen-Symbole
\usepackage{minted}  % Für Python-Code
\usepackage{xcolor}  % Zum Definieren und Verwenden von Farben
\definecolor{LightGray}{gray}{0.9}
\definecolor{abbrev}{RGB}{255,153,153}      % Rot
\definecolor{add}{RGB}{204,255,238}         % Hellgrün/Türkis
\definecolor{sic}{RGB}{255,255,153}         % Hellgelb
\definecolor{unclear}{RGB}{255,230,184}     % Hellorange
\definecolor{date}{RGB}{153,153,255}        % Blau
\definecolor{organization}{RGB}{255,153,255}% Pink
\definecolor{place}{RGB}{204,153,255}       % Lila
\definecolor{person}{RGB}{153,255,153}      % Hellgrün
\definecolor{signature}{RGB}{153,255,153}   % Hellgrün (gleiche Farbe wie Person)

\usepackage[hidelinks]{hyperref} % Für die Hyperlinks im Inhaltsverzeichnis
\usepackage[a4paper, left=2.5cm, right=2.5cm, top=2cm, bottom=2cm]{geometry}

\title{\Huge Masterarbeit Tagebuch}
\author{Sven Burkhardt}
\date{}

\begin{document}
\maketitle
\noindent
\hrulefill

\newpage % Neue Seite für das Inhaltsverzeichnis
\tableofcontents

%dieses Dokument wird blogartig geführt. Neuste Einträge sind oben einzufügen. 

%___________________________________________________________________________
% Vorlage 
\noindent\hrulefill


\subsection*{Kurzbeschreibung}
Hier kurze Zusammenfassung einfügen.

\subsection*{Erledigte Aufgaben}
\subsubsection*{\small Aufgabentitel}
\ding{51} Aufgabenbeschreibung.\\
\textbf{Ergebnis}: Ergebnisbeschreibung.

\subsection*{Nächste Schritte}
\ding{113} Beschrieb der nächsten Aufgabe.

\subsection*{Offene Fragen}
\ding{113} Beschreibung der offenen Frage.\\
%___________________________________________________________________________
% Vorlage 
\noindent\hrulefill

\section{Objektorientierte Umstrukturierung der Datenextraktions-Pipeline \s
small 08.04.2025}
\small\textit{Verbesserung des Skripts auf Basis der Empfehlungen von Claude Code}\\
\subsection*{Kurzbeschreibung}

Die bestehende Datenextraktionspipeline wurde grundlegend überarbeitet und auf eine objektorientierte Struktur umgestellt. Statt der bisherigen Dictionary-basierten Implementierung nutzen wir nun die in `document\_schemas.py` definierten Klassen wie `BaseDocument`, `Person`, `Organization` und `Place`. Diese Umstellung bringt erhebliche Vorteile bei der Datenvalidierung und -konsistenz mit sich.

\subsection*{Erledigte Aufgaben}
\subsubsection*{\small Objektorientierte Umstrukturierung von transkribus\_to\_base\_schema.py}
\ding{51} Umstellung der Datenstrukturen auf Klassen statt Dictionaries.\\
\ding{51} Integration der Validierungsfunktionen mit Fehlerausgabe.\\
\ding{51} Anpassung der JSON-Serialisierung mittels to\_json()-Methode.\\
\textbf{Ergebnis}: Robusteres Skript mit automatischer Datenvalidierung und besserer Wartbarkeit.

\subsection*{Details zur Umstrukturierung}

\begin{minted}{python}
# Alter Ansatz:
result = create_base_schema(metadata_info, transcript_text)

# Neuer objektorientierter Ansatz:
from document_schemas import BaseDocument, Person, Place, Event, Organization

doc = BaseDocument(
    attributes=metadata_info,
    content_transcription=transcript_text,
    mentioned_persons=[Person(**p) for p in custom_data["persons"]],
    mentioned_organizations=[Organization(**o) for o in custom_data["organizations"]],
    mentioned_places=[Place(**pl) for pl in custom_data["places"]],
    mentioned_dates=custom_data["dates"]
)

# Mit Validierung
if not doc.is_valid():
    print(f"Fehler in Dokument: {doc.validate()}")
\end{minted}

\subsection*{Vorteile der neuen Implementierung}
\begin{itemize}
    \item \textbf{Datenvalidierung:} Automatische Überprüfung auf strukturelle Korrektheit
    \item \textbf{Typsicherheit:} Verbesserte Erkennung von Fehlern bereits zur Programmierzeit
    \item \textbf{Objektorientierung:} Zusammengehörige Daten und Funktionen in einer Klasse
    \item \textbf{Bessere Wartbarkeit:} Klare Schnittstellen und Kapselung von Implementierungsdetails
    \item \textbf{Erweiterbarkeit:} Einfachere Anpassung an neue Anforderungen durch Vererbung
\end{itemize}

\subsection*{Nächste Schritte}
\ding{113} Überprüfung aller generierten JSON-Dateien auf Validität.\\
\ding{113} Integration von fortgeschrittenen Informationsextraktionsmethoden für Entitäten.\\
\ding{113} Automatisierte Tests für die Datenverarbeitung entwickeln.

\subsection*{Offene Fragen}
\ding{113} Wie können wir die Qualität der Entitätserkennung verbessern?\\
\ding{113} Sollen zusätzliche spezialisierte Dokumentklassen implementiert werden?

\section{Implementierung der Datenextraktions-Pipeline \small 03.04.2025}
\small\textit{{Diese Implementierung wurde mit Unterstützung von Claude Code und Sorin von RISE durchgeführt. Erster Einsatz von Claude Code}}\\
\subsection*{Kurzbeschreibung}

Implementierung der Datenkonvertierungspipeline vom Transkribus XML-Format in das projektspezifische JSON-Basisschema. Es wurden zwei Python-Module entwickelt: ein Schema-Modul und ein Konvertierungsmodul, die zusammen die Grundlage für die strukturierte Datenhaltung bilden.

\subsection*{Erledigte Aufgaben}
\subsubsection*{\small Definition des Basis-Schemas}
\ding{51} Erstellung des \textbf{\textit{document\_schemas.py}} Moduls mit Klassen für verschiedene Dokumenttypen.\\
\textbf{Ergebnis}: Implementierung der Basisklassen \texttt{Person}, \texttt{Organization}, \texttt{Place}, \texttt{Event} und \texttt{BaseDocument} sowie spezialisierte Klassen für \texttt{Brief}, \texttt{Postkarte} und \texttt{Protokoll} mit Validierungsfunktionen.

\subsubsection*{\small Konvertierungs-Pipeline}
\ding{51} Implementierung des \textbf{\textit{transkribus\_to\_base\_schema.py}} Skripts für die XML-Verarbeitung.\\
\textbf{Ergebnis}: Vollständige Pipeline für die Extraktion von Metadaten und Text aus Transkribus XML-Dateien sowie Konvertierung in das JSON-Basisschema mit automatischer Erkennung von benannten Entitäten (Personen, Organisationen, Orte, Daten).

\subsection*{Nächste Schritte}
\ding{113} Integration der Tag-Regeln aus der Transkribus-Anleitung in die Konvertierungspipeline. \\
\ding{113} Erweitern der automatischen Metadatenextraktion für komplexere Inhaltsanalysen.\\
\ding{113} Ausführen der Pipeline auf dem vollständigen Datensatz und Prüfung der Ergebnisse.\\

\subsection*{Offene Fragen}
\ding{113} Sollen zusätzliche spezialisierte Dokumenttypen implementiert werden?\\
\ding{113} Wie können Entitätsreferenzen zwischen verschiedenen Dokumenten hergestellt werden?

\textit{Diese Implementierung wurde mit Unterstützung von Claude Code und Sorin von RISE durchgeführt.}

\noindent\hrulefill
\section{Festlegung Tagging Regeln \small 11.03.2025 } % Kurztitel

\subsection*{Kurzbeschreibung}
Hier werden die Tagging Regeln in Transkribus beschrieben
 
\section*{Personen \texttt{\texttt{\textbf{{\colorbox{person}{person}}}}}}

Mit dem Tag \texttt{\texttt{\texttt{\textbf{{\colorbox{person}{person}}}}}} sollen alle strings getaggt werden, die eine direkte Zuordnung eier Person ermöglichen. \\

\noindent\ding{43} Beispiel 1: Vereinsführer, Alfons, Zimmermann, Alfons Zimmermann, Z. A.Zimmermann, Herr Zimmermann, Herr Alfons Zimmermann, etc \\
\ding{43} Beispiel 2: Gauleitung, Gauleiter, Gauleiter Max Mustermann, Gauleiter M., Gauleiter Mustermann, etc\\
\ding{43} Beispiel 3: Oberlehrer, Chorleiter, etc, wenn Ort, Name oder Organisation bekannt. \\

\section*{Signaturen \texttt{\texttt{\textbf{{\colorbox{signature}{signature}}}}}}

Mit dem Tag \texttt{\texttt{\texttt{\textbf{{\colorbox{signature}{signature}}}}}} sollen alle Strings getaggt werden, die eine handschriftliche Unterschrift darstellen. Dieses Tag dient dazu, Unterschriften von klar erkennbaren Personennamen im Fließtext zu unterscheiden. \\

\noindent\ding{43} Beispiel 1: **Eindeutig lesbare Signaturen** werden direkt getaggt:  
\texttt{\textbf{\textless signature\textgreater R. Weiss\textless /signature\textgreater}}.\\  

\ding{43} Beispiel 2: **Teilweise unleserliche Signaturen** werden mit dem Tag \texttt{unclear} innerhalb von \texttt{signature} markiert:  
\texttt{\textbf{\textless signature\textgreater R. We\textless unclear\textgreater [...]\textless /unclear\textgreater\textless /signature\textgreater}}.\\  

\ding{43} Beispiel 3: **Wenn nur ein Teil des Namens lesbar ist, aber eine Identifikation unsicher bleibt**, sollte die Unterschrift vollständig im Tag \texttt{unclear} innerhalb von \texttt{signature} stehen:  
\texttt{\textbf{\textless signature\textgreater \textless unclear\textgreater Unleserlich\textless /unclear\textgreater \textless /signature\textgreater}}.\\  

\ding{43} Beispiel 4: **Wenn eine Signatur einer bekannten Person zugeordnet werden kann, aber nicht vollständig lesbar ist, bleibt die Signatur erhalten und wird nicht als „Person“ getaggt**:  
\texttt{\textbf{\textless signature\textgreater A. Zimm\textless unclear\textgreater [...]\textless /unclear\textgreater\textless /signature\textgreater}}.\\  

\section*{Organisationen \texttt{\texttt{\textbf{{\colorbox{organization}{organization}}}}}}


Mit dem Tag \texttt{\texttt{\textbf{{\colorbox{organization}{organization}}}}} sollen alle Strings getaggt werden, die eine direkte Zuordnung einer Organisation ermöglichen. \\

\noindent\ding{43} Beispiel 1: Männerchor Murg, Verein Deutscher Arbeiter (V.D.A.), Murgtalschule, etc.\\
\ding{43} Beispiel 2: Abkürzungen, wenn sie eine Organisation eindeutig bezeichnen, z.B. V.D.A., NSDAP, STAGMA, etc.\\
\ding{43} Beispiel 3: Wenn der Organisationsname mit einer Positionsangabe kombiniert ist, wird nur der Organisationsname getaggt: „\textless person\textgreater Vereinsleiter \textless person\textgreater des \textbf{\textless organisation\textgreater Männerchor Murg\textless /organisation\textgreater}“. \\

\section*{Orte \texttt{\texttt{\textbf{{\colorbox{place}{place}}}}}}


Mit dem Tag\texttt{\texttt{\textbf{{\colorbox{place}{place}}}}} sollen alle Strings getaggt werden, die sich auf einen geografischen Ort beziehen. \\
\indent\textit{Hintergund ist Fehlervermeidung: Manche Organisationen haben mehrere Standorte (z. B. "Badischer Sängergau" kann in Karlsruhe oder Mannheim sein). Ein explizites Tagging vermeidet Verwechslungen.
Flexibilität: Falls sich eine Organisation an mehreren Orten befindet oder in einem bestimmten Kontext mit einem anderen Ort assoziiert wird, bleibt die Information strukturiert erhalten.} \\

\noindent\ding{43} Beispiel 1: Murg (Baden), Freiburg, Berlin, Murgtal, Schwarzwald, etc.\\
\ding{43} Beispiel 2: Orte mit näherer Bestimmung, z.B. „bei Berlin“, „im Murgtal“, "Mayerhof Nebenzimmer" werden getaggt, und die nähere Bestimmung muss innerhalb des Tags. Beispiel: „ \textbf{\textless place\textgreater  im Murgtal\textless /place\textgreater }“.\\
\ding{43} Beispiel 3: Adressen oder Ortsangaben in Verbindung mit Organisationen, z.B. „\textless organisation\textgreater Universität \textbf{\textless place\textgreater  Basel\textless /place\textgreater} \textless /organisation\textgreater“ oder „Postfach 6, \textbf{\textless place\textgreater  Murg \textless /place\textgreater }“. \\

\section*{Datum \texttt{\texttt{\textbf{{\colorbox{date}{date}}}}}}


Mit dem Tag \texttt{\texttt{\texttt{\textbf{{\colorbox{date}{date}}}}}} werden alle expliziten Datumsangaben markiert und bereits im Tag um das format dd.mm.yyyy ergänzt.\\

\noindent\ding{43} Beispiel 1: 9. Oktober 1940, 20.10.1940, den 3. Mai 1938, etc.\\
\ding{43} Beispiel 2: Relative Datumsangaben („gestern“, „letzten Freitag“) werden getaggt.\\
\ding{43} Beispiel 3: Falls ein Datum in Kombination mit einem Ort steht, wird nur das Datum getaggt: „\textbf{\textless place\textgreater Murg (Baden)\textless /place\textgreater }, den \textbf{\textless date\textgreater 9. Oktober 1940\textless /date\textgreater }“. \\

\section*{Abkürzungen \texttt{\textbf{{\colorbox{abbrev}{abbrev}}}}}


Mit dem Tag \texttt{\texttt{\textbf{{\colorbox{abbrev}{abbrev}}}}} werden alle Abkürzungen getaggt, die für eine eindeutige Entität stehen. \\

\noindent\ding{43} Beispiel 1: Dr., Prof., St., Hr., Frl., Dipl.-Ing., etc.\\
\ding{43} Beispiel 2: Organisationskürzel, wenn sie eindeutig sind: „\textbf{\textless abbrev\textgreater V.D.A.\textless /abbrev\textgreater }“.\\
\ding{43} Beispiel 3: Falls eine ausgeschriebene Variante im selben Dokument vorhanden ist, bleibt die Abkürzung getaggt: \textbf{\textless person\textgreater \textless abbrev\textgreater Dr.\textless /abbrev\textgreater Weiß\textless /person\textgreater}


\section*{Unclear\texttt{\textbf{{\colorbox{unclear}{unclear}}}}}


Mit dem Tag \texttt{\texttt{\textbf{{\colorbox{unclear}{unclear}}}}} werden unleserliche oder schwer entzifferbare Textstellen markiert. \\

\noindent\ding{43} Beispiel 1: Unklare Zeichen oder fehlende Buchstaben: „Er wohnte in \textbf{\textless unclear\textgreater [...]\textless unclear\textgreater }“.\\
\ding{43} Beispiel 2: Teilweise lesbare Wörter: „\textbf{\textless place\textgreater Frei\textless unclear\textgreater [...]\textless unclear\textgreater \textless place\textgreater }“.\\


\section*{Sic \texttt{\textbf{{\colorbox{sic}{sic}}}}}


Mit dem Tag \texttt{\texttt{\textbf{{\colorbox{sic}{sic}}}}} werden Wörter markiert, die absichtlich in einer falschen oder ungewöhnlichen Schreibweise beibehalten werden. \\

\noindent\ding{43} Beispiel 1: Offensichtliche Tippfehler, wenn sie im Originaltext so vorkommen: „Er hatt \textbf{\textless sic\textgreater einen\textless /sic\textgreater } große Freude.“\\
\ding{43} Beispiel 2: Veraltete oder falsche Schreibweisen: „\textbf{\textless sic\textgreater Feber\textless /sic\textgreater }“ für Februar.\\
\ding{43} Beispiel 3: Falls eine Korrektur notwendig ist, kann sie als Kommentar ergänzt werden. \\


%___________________________________________________________________________
\newpage % Nach dem Inhaltsverzeichnis eine neue Seite


\section{Überblick Hilfscript\\CHatGPT\_Api\_TranskribusXML\_to\_JSONv3.py \\\small 06.03.2025}
Dieses Skript verarbeitet XML-Dateien aus einer definierten Ordnerstruktur und sendet deren Inhalt an die OpenAI-API zur automatischen Analyse und Extraktion relevanter Metadaten. Das generierte Ergebnis wird als JSON-Datei gespeichert.

Das Skript ist speziell auf historische Dokumente (z. B. aus dem Männerchor Murg Corpus, 1925-1945) zugeschnitten. Es analysiert und strukturiert die Daten, um relevante Informationen wie Autoren, Empfänger, Orte, Ereignisse und Zeitangaben zu extrahieren.

\section{Voraussetzungen \& Einrichtung}

\subsection{API Key setzen}
Bevor das Skript ausgeführt wird, muss der OpenAI-API-Schlüssel als Umgebungsvariable gesetzt werden. Dies kann dauerhaft in der ~/.zshrc-Datei erfolgen:

\begin{minted}{bash}
    export OPENAI\_API\_KEY=''sk-...''
    source ~/.zshrc
\end{minted}    

Alternativ kann das Skript mit VS Code gestartet werden, indem es aus dem Terminal mit folgendem Befehl aufgerufen wird:

\begin{minted}{bash}
code .
\end{minted}

\subsection{Python-Bibliotheken installieren}
Folgende Bibliotheken werden benötigt:

\begin{minted}{bash}
pip install openai
\end{minted}

folgende Module werden gebraucht:

\begin{minted}{bash}
    import json
    import os
    import xml.etree.ElementTree as ET
    import openai
    import re
    import time
\end{minted}

\section{Funktionsweise}

Das Skript ist in folgende Hauptbestandteile gegliedert:

\subsection{Basis-Einstellungen}
\begin{itemize}
\item Zähler zur Erfassung der verarbeiteten Dateien und Token-Kosten.
\item API-Schlüssel wird aus der Umgebungsvariable geladen.
\item Pfad-Definitionen für Input- und Output-Verzeichnisse.
\end{itemize}

\subsection{Verzeichnisstruktur durchlaufen}
Das Skript iteriert über alle 7-stelligen Ordner im Basisverzeichnis (base\_input\_directory). In diesen Ordnern sucht es nach Unterordnern mit Namen Akte\_xxx\_pdf oder Akte\_xxx, die wiederum einen page-Ordner enthalten.

Falls kein page-Ordner gefunden wird, wird die Verarbeitung dieser Akte übersprungen.

\subsection{Verarbeitung der XML-Dateien}
Jede XML-Datei im page-Ordner wird eingelesen und analysiert:
\begin{itemize}
\item Seitennummer extrahieren: Die Dateinamen haben das Muster p001.xml, p002.xml usw.
\item XML-Daten einlesen: Nutzung von xml.etree.ElementTree zur Extraktion von TranskribusMetadata und TextEquiv-Daten.
\item Fehlermanagement: Falls eine Datei nicht geparst werden kann oder keinen verwertbaren Text enthält, wird sie übersprungen.
\end{itemize}

\subsection{Strukturierung der Daten in JSON}
Die extrahierten Informationen werden in einem JSON-Format gespeichert. Dazu gehören:
\begin{itemize}
\item Metadaten (Dokument-ID, Seiten-ID, Bild- und XML-URL)
\item Autor und Empfänger mit Name, Rolle und zugehöriger Organisation
\item Erwähnte Personen, Organisationen, Ereignisse und Orte
\item Dokumentart (z. B. Brief, Protokoll, Rechnung)
\item Dokumentformat (z. B. Handschrift, maschinell, mit Unterschrift, Bild)
\end{itemize}

\subsection{API-Anfrage an OpenAI}
Ein Prompt wird erstellt, um die Inhalte durch die OpenAI-API analysieren zu lassen. Dabei wird explizit vorgegeben:
\begin{itemize}
\item Die historische Relevanz der Dokumente (1925-1945).
\item Die Aufgabenstellung (Identifikation von Dokumenttypen, Metadaten und Inhalten).
\item Die genaue Formatierung der JSON-Antwort.
\end{itemize}

\section{Fehlerbehandlung}
Das Skript enthält mehrere Mechanismen zur Fehlerbehandlung:
\begin{itemize}
\item Fehlermeldungen beim XML-Einlesen (try-except beim Parsen)
\item Fehlermeldungen bei API-Anfragen (try-except um den OpenAI-Aufruf)
\item Fehlermeldungen bei JSON-Speicherung (try-except beim Schreiben der Datei)
\item Logging von Problemen (print(f''Fehler beim Parsen der API-Antwort: {e} ''))
\end{itemize}

Falls Fehler auftreten, werden sie ausgegeben und das Skript setzt die Verarbeitung der nächsten Datei fort, anstatt komplett abzubrechen.

\section{Fazit}
Dieses Skript automatisiert die Verarbeitung von XML-Dokumenten, extrahiert deren Inhalte und strukturiert die Daten in JSON-Format, das von OpenAI-API analysiert wird. Die Ergebnisse werden gespeichert und abschließend statistisch ausgewertet. Durch Fehlerbehandlung und Logging wird sichergestellt, dass auch bei Problemen das Skript robust bleibt. Dokumentation geschrieben mit ChatGPT.

%___________________________________________________________________________
\noindent\hrulefill
\section{Tagging der JPGEGs im AppleFinder \small 26.10.24 } % Kurztitel

\subsection*{Kurzbeschreibung}

Überlegung:
JPEGs sollen bereits im Apple Finder mit Tags versehen werden, um eine effiziente, automatisierte Transkription der Chorunterlagen des Männerchors Murg zu ermöglichen. Geplant ist die Kombination von ChatGPT und Transkribus zur Erkennung unterschiedlicher Dokumententypen. Ein Tag-System, bestehend aus \textbf{„Maschinell“} für maschinengeschriebene und \textbf{„Handschrift“} für handschriftliche Dokumente, gewährleistet die gezielte Zuordnung zur jeweils geeigneten OCR-Software \textit{(Maschinenschrift mit ChatGPT, Handschrift mit Transkribus ''German Giant'')}.

Dokumente, die sowohl maschinell erstellten Text als auch handschriftliche Elemente enthalten, werden entsprechend ihrer Hauptinformationsgehalt getaggt. Zusätzlich erhalten alle Dokumente mit Unterschriften den Tag \textbf{„Unterschrift“}, um eine gezielte Verarbeitung dieser Elemente sicherzustellen.

\subsection{AppleFinder Tags}

\textcolor{yellow}{\Large $\bullet$} \textbf{Handschrift} \\
\textcolor{blue}{\Large $\bullet$} \textbf{Maschinell} \\
\textcolor{gray}{\Large $\bullet$} \textbf{mitUnterschrift} \\
\textcolor{purple}{\Large $\bullet$} \textbf{Bild} \\

\subsection*{Erledigte Aufgaben}
\subsubsection*{\small Handschriften tagging}
\ding{51} taggen \textcolor{yellow}{\Large $\bullet$} \textbf{Handschrift} in AppleFinder. \\
\ding{51} taggen \textcolor{blue}{\Large $\bullet$} \textbf{Maschinell} in AppleFinder. \\
\ding{51} taggen \textcolor{purple}{\Large $\bullet$} \textbf{Bild} \\
\ding{51} taggen \textcolor{gray}{\Large $\bullet$} \textbf{mitUnterschrift} \\
\\
\textbf{Ergebnis}: Handschriften, Maschinell, Bilder und alle handschriftlichen Unterschriften getaggt

\subsection*{Nächste Schritte}
\ding{113} Skripte schreiben, um maschinelle Text zu extrahieren 
\ding{113} Transkribus für Handschriftliches anschmeißen. \\
\ding{113} Nach Gemeinsamkeiten in den Texten suchen, um automatisierte Abfrage für ChatGPT zu erstellen. \\ 
\ding{113} Ggf. Aufteilung in unterschiedliche Korpora (Briefe handschr. Briefe schreibmaschine, Zeitungsunterlagen.)\\
\ding{113} Transkribus für Handschriften verwenden. \\

\subsection*{Offene Fragen}
\ding{113} Sollen die Bilder gelöscht werden?\\
%-------------------------------------------------------


%___________________________________________________________________________

\noindent\hrulefill
\section{JPG Datenbereinigung - leere Seiten löschen \small 25.10.2024 } % Kurztitel

\subsection*{JPG Datenbereinigung}
Alle JPGs ohne Inhalt, also beispielsweise Rückseiten, werden gelöscht. Regel: sobald etwas handschriftlich oder gedruckt auf einer Seite steht, bleibt es erhalten. Im Moment sind auch Bilder (Bsp. Postkarten inbegriffen.Bilder mit Taggs 
\subsection{Anmerkung}
Geschichte/Chronik/Gründung des Männerchors in Akte 323

\subsection*{Erledigte Aufgaben}
\subsubsection*{\small JPG Datenbereinigung }
\ding{51}Alle JPGs ohne Inhalt, also beispielsweise Rückseiten, werden gelöscht.\\
\textbf{Ergebnis}: Reiner JPG Korpus mit Schriftgut, aber auch Bildern (bspw. Postkarten)

\subsection*{Nächste Schritte}
\ding{113} Nach Gemeinsamkeiten in den Texten suchen, um automatisierte Abfrage für ChatGPT zu erstellen. \\ 
\ding{113}Ggf. Aufteilung in unterschiedliche Korpora (Briefe handschr. Briefe schreibmaschine, Zeitungsunterlagen.)\\
\ding{113} Transkribus für Handschriften verwenden.

\subsection*{Offene Fragen}
\ding{113} Sollen die Bilder gelöscht werden?\\
\ding{113} Handschriftliche, maschinengeschriebene und gemischte Daten taggen? Ggf. erst später mit ChatGPT.\\
\ding{113} Transkribus für Handschriften verwenden?

%___________________________________________________________________________

\noindent\hrulefill

\section{Datennormalisierung PDF zu JPEG \small 24.10.2024}
\subsection*{Kurzbeschrieb}
Heute zwei Python-Skripte zur Normalisierung der Akten geschrieben:

\subsection*{Erledigte Augaben}
\subsubsection*{\small PDF zu JPG Konvertierung}
\ding{51} Skript \textbf{\textit{JPEG-to-PDF.py}} geschrieben.  \\
\textbf{Ergebnis} Alle PDF-Seiten in JPGs umgewandelt, Dateinamen mit Seitenzahlen formatiert.

\subsubsection*{\small Prüfung der Aktennummern}
\ding{51} Skript \textbf{\textit{Check-if-all-files-complete.py}} geschrieben. \\
\textbf{Ergebnis}: Überprüft, ob Akten von 001 bis 425 vorhanden sind. Alle Akten sind vollständig in JPG umgewandelt.

\subsection*{Nächste Schritte}
\ding{113} Daten für OCR-Bereinigung vorbereiten, leere Seiten manuell entfernen.
\subsection*{Offene Fragen}
\ding{113} Handschriftliche, maschinengeschriebene und gemischte Daten taggen? Ggf. erst später mit ChatGPT.

\noindent\hrulefill

%___________________________________________________________________________
\end{document}

